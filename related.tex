\section{Related Work}
\label{sec:related}

\balance
Mergeable Replicated Data Types (MRDTs) were introduced
in~\cite{mrdt}, where authors also demonstrate an approach for
deriving merge functions from first principles.  Despite resulting in
sensible merge semantics for several data types, their approach was
never shown to guarantee the convergence of the resultant MRDTs.
Indeed, as we demonstrate in Sec.~\ref{sec:motivation}, convergence
of unrestricted MRDT executions cannot be guaranteed due to the
presence of anomalies such as Fig.~\ref{fig:mrdt-exec-2} and
Fig.~\ref{fig:mrdt-exec-4}. \quark fixes this problem by extending
MRDTs with a runtime that limits their executions to those that are
guaranteed to converge.

% Concurrent Revisions~\cite{BBL+10} proposes versioned state for
% shared-memory concurrent programs in order to guarantee determinism.
% The fork-join concurrency they consider is however too restrictive for
% the asynchronous distributed setting where replicas do not have a
% single coherent view of the system state.  Furthermore the
% hierarchical organization of the concurrent actors with a designated
% \emph{server} is inconsistent with the fully-decentralized model of
% execution adopted by several collaborative applications.

State-centric replication was also explored in the context of
CRDTs~\cite{crdts}. However, such state-based CRDTs require the
replicated state to be organized as a lattice with the merge function
acting as as a least upper bound operator. We eliminate this
restriction in our setting with help of the \quark runtime. A thorough
comparison of our approach with the operation-based CRDTs can be found
in Secs.~\ref{sec:introduction} and~\ref{sec:motivation}.

Several verification techniques, program analyses, and tools have been
proposed to reason about the program behavior in a weakly-consistent
distributed setting~\cite{bailis-vldb, alvaro-calm,
gotsman-popl16, pldi15, redblue-osdi}. These techniques
treat replicated storage as a black box with a fixed pre-defined
consistency models. The focus is on assigning appropriate consistency
levels to operations so as to preserve application integrity. Such an
approach results in assigning sequential consistency (SC) to all
operations since the next weaker consistency model -- causal
consistency, is insufficient to guarantee convergence (as
Sec.~\ref{sec:motivation} demonstrates). Conversely, \quark cannot
reason about application-level invariants, such as $\C{balance}\ge 0$
in a banking application. Thus both approaches confer complimentary
benefits on application developers.

% Type systems
The meta-theory of \quark abstract machine
(Sec.~\ref{sec:abstract-sem} bears resemblance to the type safety
proofs carried out in Wright and Felleisen's style~\cite{WF92}.
Theorem~\ref{thm:progress} (Progress) is analogous to the Progress
lemma of type safety, and Theorem~\ref{thm:convergence} (Convergence)
is analogous to Preservation.  However, unlike the type-based
approaches, safety in \quark is enforced through run-time monitoring.
Static enforcement of convergence in MRDTs using, for e.g., Session
Types~\cite{HNMSession16}, is an interesting direction for future
work.

% Git
\quark's programming model is inspired by distributed version control
systems in general, and Git in particular. To the best of our
knowledge, operational semantics of Git has never been formalized.
This is not a serious concern considering that humans are heavily
involved in Git workflows, and the complexity of version histories
manifest by Git is bound by the limits of human cognition.
Nonetheless, it is possible to construct Git version histories that
result in non-convergent and counter-intuitive final
states~\cite{tycon,russell}. Such anomalies are more likely to occur
if Git version histories are manifest commensurately with distributed
executions of computer programs. \quark's run-time monitoring
pre-empts this possibility in our case.

Finally, the implementation of \quark bears resemblance to
Quelea~\cite{pldi15} as both systems offload low-level concerns to an
underlying data store. While Quelea enforces high-level invariants on
applications built with CRDTs that are assumed to be convergent,
\quark enforces convergence of data types that are otherwise not
guaranteed to converge.
