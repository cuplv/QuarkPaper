%% For double-blind review submission, w/o CCS and ACM Reference (max submission space)
\documentclass[sigplan,screen]{acmart}
%\settopmatter{printfolios=true,printccs=false,printacmref=false}
%% For final camera-ready submission, w/ required CCS and ACM Reference
%\documentclass[acmsmall]{acmart}\settopmatter{}

%\usepackage{amssymb}
\usepackage{amsmath}
\usepackage{amsfonts}
\usepackage{caption}
\usepackage{subcaption}
\usepackage{xspace}
\usepackage{mathtools}
\usepackage{mathpartir}
\usepackage{ifpdf}
\usepackage{graphicx}
%\usepackage[usenames,dvipsnames]{color}
\usepackage{stmaryrd}
%\usepackage[numbers]{natbib}
\usepackage{amsthm}
\usepackage{listings}          % format code
\usepackage{wrapfig}
\usepackage{textcomp}
\usepackage{tabularx}
\usepackage{color}
\usepackage{url}
\usepackage{tikz}
\usepackage{multirow,array}
\usepackage[utf8]{inputenc}
\usepackage[T1]{fontenc}
\usepackage{microtype}

% Math mode
%-----------
\newenvironment{nop}{}{}
\newenvironment{smathpar}{
\begin{nop}\small\begin{mathpar}}{
\end{mathpar}\end{nop}\ignorespacesafterend}
\newcounter{hypothesis}
\newenvironment{hypothesis}{\refstepcounter{hypothesis}}{}

% Theorem
%--------

\theoremstyle{plain}
\newtheorem{axiom}{Axiom}[section]
\newtheorem{theorem}{Theorem}[section]
\newtheorem{lemma}[theorem]{Lemma}
\newtheorem{proposition}[theorem]{Proposition}
\newtheorem{corollary}[theorem]{Corollary}
\theoremstyle{definition}
\newtheorem{definition}[theorem]{Definition}

\newenvironment{example}[1][Example]{\begin{trivlist}
\item[\hskip \labelsep {\bfseries #1}]}{\end{trivlist}}
\newenvironment{remark}[1][Remark]{\begin{trivlist}
\item[\hskip \labelsep {\bfseries #1}]}{\end{trivlist}}

% Decorations
%-----------
\newenvironment{decoration}
  {\color{blue}\begin{array}{l}}
  {\end{array}}

% New colors
%------------
\definecolor{Bittersweet}{rgb}{1.0, 0.44, 0.37}
\definecolor{MidnightBlue}{rgb}{0.0, 0.2, 0.4}
\definecolor{BrightBlue}{rgb}{0.0, 0.2, 0.7}

% Listings
%----------
\newcommand{\lstml}{
\lstset{ %
language=ML, % choose the language of the code
basicstyle=\footnotesize\ttfamily,       % the size of the fonts that are used for the code
keywordstyle=\color{Bittersweet},
% numbers=left,                   % where to put the line-numbers
numberstyle=\tiny,      % the size of the fonts that are used for the line-numbers
stepnumber=1,                   % the step between two line-numbers. If it is 1 each line will be numbered
numbersep=5pt,                  % how far the line-numbers are from the code
showspaces=false,               % show spaces adding particular underscores
showstringspaces=false,         % underline spaces within strings
showtabs=false,                 % show tabs within strings adding particular underscores
% frame=single,                   % adds a frame around the code
tabsize=2,                      % sets default tabsize to 2 spaces
captionpos=b,                   % sets the caption-position to bottom
breaklines=true,                % sets automatic line breaking
breakatwhitespace=false,        % sets if automatic breaks should only happen at whitespace
commentstyle=\itshape\color{BrightBlue},
%escapeinside={\%*}{*)},         % if you want to add a comment within your code
mathescape=true,
morekeywords={oper, txn, invariant, module, begin, match, when, @@deriving, not, : , txn_do, do, SQL/\\}
}}
\lstnewenvironment{ocaml}
    { % \centering
			\lstml
      \lstset{}%
      \csname lst@setfirstlabel\endcsname}
    { %\centering
      \csname lst@savefirstlabel\endcsname}
\newcommand{\ocamlinline}[1]{\lstinline[language=ML,
                                        basicstyle=\footnotesize\ttfamily, 
                                        keywordstyle=\color{Bittersweet},
                                        mathescape=true]!#1!}

% SQL Trace 
% ----------
\newcommand{\lstsql}{
\lstset{ %
  language=SQL, % choose the language of the code
  basicstyle=\footnotesize\ttfamily,       % the size of the fonts that are used for the code
  keywordstyle=\color{MidnightBlue},
  % numbers=left,                   % where to put the line-numbers
  numberstyle=\tiny,      % the size of the fonts that are used for the line-numbers
  stepnumber=1,                   % the step between two line-numbers. If it is 1 each line will be numbered
  numbersep=5pt,                  % how far the line-numbers are from the code
  showspaces=false,               % show spaces adding particular underscores
  showstringspaces=false,         % underline spaces within strings
  showtabs=false,                 % show tabs within strings adding particular underscores
  % frame=single,                   % adds a frame around the code
  tabsize=2,                      % sets default tabsize to 2 spaces
  captionpos=b,                   % sets the caption-position to bottom
  breaklines=true,                % sets automatic line breaking
  breakatwhitespace=false,        % sets if automatic breaks should only happen at whitespace
  commentstyle=\itshape\color{BrightBlue},
  %escapeinside={\%*}{*)},         % if you want to add a comment within your code
  mathescape=true,
  morekeywords={BEGIN, COMMIT, ROLLBACK}
}}
\lstnewenvironment{sqltrace}
    { % \centering
			\lstsql
      \lstset{}%
      \csname lst@setfirstlabel\endcsname}
    { %\centering
      \csname lst@savefirstlabel\endcsname}

\newcommand{\sql}[1]{\lstinline[language=SQL,
                                basicstyle=\footnotesize\ttfamily, 
                                keywordstyle=\color{BrightBlue},
                                breaklines=true,
                                breakatwhitespace=false,
                                mathescape=true,
                                morekeywords={BEGIN, COMMIT, ROLLBACK}]!#1!}


% Formatting
%---------
\newcommand{\C}[1]{\code{#1}}
\newcommand{\tuplee}[1]{\langle #1 \rangle}
\newcommand*{\rom}[1]{\expandafter\romannumeral #1}

% Formatting commands
% -------------------
\newcommand{\code}[1]{{\tt #1}}
\newcommand{\spc}[0]{\quad}
\newcommand{\ALT}{~\mid~}
\newcommand{\rel}[1]{{R}_{\mathit{#1}}}
\newcommand{\conj}{~\wedge~}
\newcommand{\disj}{~\vee~}
\newcommand{\rulelabel}[1]{\textrm{\sc {#1}}}
\newcommand{\ilrulelabel}[1]{{\sc #1}}
\newcommand{\RULE}[2]{\frac{\begin{array}{c}#1\end{array}}
                           {\begin{array}{c}#2\end{array}}}
\newcommand{\txnimp}{\mbox{${\cal T}$}}
\newcommand{\ssnimp}{{\sc SsnImp}\xspace}
%\newcommand{\coloneqq}{::=}
\newcommand{\qqquad}{\quad\quad}
\newcommand{\cskip}{\C{SKIP}}
\newcommand{\ctxn}[2]{\C{TXN}\langle #1 \rangle\{#2\}}
\newcommand{\csess}[2]{\C{ssn}\langle #1 \rangle \{#2\}}
\newcommand{\catomic}[1]{\C{ATOMIC}\{#1\}}
\newcommand{\stepsto}{\longrightarrow}
\newcommand{\stepssto}[1]{\longrightarrow^{#1}_{R}}
\newcommand{\xstepsto}[1]{\longrightarrow_{#1}}
\newcommand{\xstepssto}[2]{\longrightarrow^{#1}_{#2}}
\newcommand{\tstepsto}{\longrightarrow}
\newcommand{\redsto}{\hookrightarrow}
\newcommand{\xtstepsto}[1]{\hookrightarrow_{#1}}
\newcommand{\rtstepsto}{\hookrightarrow_R}
\newcommand{\rtstepssto}[1]{\hookrightarrow^{#1}_R}
\newcommand{\xtstepssto}[2]{\hookrightarrow^{#1}_{#2}}
\newcommand{\hoare}[3]{\{#1\}\,#2\,\{#3\}}
\newcommand{\defeq}[0]{\overset { \mathit{def} }{ = } }
\newcommand{\rg}[3]{\{#1\}\,#2\,\{#3\}}
%\newcommand{\defeq}[0]{ \triangleq }
\newcommand{\op}{\textsf{op}}
\newcommand{\E}{\mathcal{E}}
\newcommand{\I}{\mathbb{I}}
\newcommand{\F}{{\sf F}}
\newcommand{\G}{{\sf G}}
\newcommand{\D}{\mathcal{D}}
\newcommand{\T}{\mathcal{T}}
\renewcommand{\P}{\mathcal{P}}
\newcommand{\Lfull}{L}
\newcommand{\Lnoloop}{L_{\diamond}}
\newcommand{\Lnoif}{L_{\downarrow}}
\newcommand{\visZ}{\textsf{vis}}
\newcommand{\soZ}{\textsf{so}}
\newcommand{\hbZ}{\textsf{hb}}
\newcommand{\sameobj}[2]{\textsf{sameobj}(#1,#2)}
\newcommand{\sameobjZ}{\textsf{sameobj}}
\newcommand{\visar}{\xrightarrow{\visZ}}
\newcommand{\hboar}{\xrightarrow{\textsf{hb}}}
\newcommand{\soar}{\xrightarrow{\soZ}}
\newcommand{\visoar}{\xrightarrow{\visZ \,\cup\, \soZ}}
\newcommand{\invisar}{\xrightarrow{\textsf{invis}}}
\newcommand{\etaar}{\xrightarrow{\eta}}
\newcommand{\wrstoar}{\xrightarrow{\textsf{wrsto}}}
\newcommand{\rdsfmar}{\xrightarrow{\textsf{rdsfm}}}
\newcommand{\usesar}{\xrightarrow{\textsf{uses}}}
\newcommand{\isReadf}{\textsf{isRD}}
\newcommand{\isWritef}{\textsf{isWR}}
\newcommand{\oper}{\textsf{oper}}
\newcommand{\committed}{\textsf{com}}
\newcommand{\txn}{\textsf{txn}}
\newcommand{\ssn}{\textsf{ssn}}
\newcommand{\id}{\textsf{id}}
\newcommand{\kind}{\textsf{oper}}
\newcommand{\obj}{\textsf{obj}}
\newcommand{\rval}{\textsf{rval}}
\newcommand{\visible}{\textsf{visible}}
\newcommand{\maxId}{\textsf{maxId}}
\newcommand{\aeval}{\textsf{aeval}}
\newcommand{\underE}[1]{\E \Vdash #1}
\newcommand{\underIT}[1]{\;\I,\C{Txn}_i \vdash #1\;}
\newcommand{\underI}[1]{\;\I \vdash #1\;}
\newcommand{\underT}[1]{\;\C{Txn}_i \vdash #1\;}
\newcommand{\stable}{\mathtt{stable}}
\newcommand{\iso}[1]{\emph{#1}}
\newcommand{\writef}{\textsf{Write}}
\newcommand{\readf}{\textsf{Read}}
\newcommand{\commitf}{\textsf{Commit}}
\newcommand{\eg}{\emph{e.g.,}}
\newcommand{\GK}[1]{\textcolor{red}{#1}}
\newcommand{\SJ}[1]{\textcolor{red}{SJ: #1}}
\newcommand{\valid}{\textsf{valid}}

\newcommand{\B}[1]{\small\bf #1}
\newcommand{\txnbox}[1]{\lbrack #1 \rbrack}
\newcommand{\Prop}{\mathbb{P}}
\newcommand{\Pow}[1]{\mathcal{P}\left(#1\right)}
\renewcommand{\bar}[1]{\overline{#1}}
\renewcommand{\merge}{\diamond}
\newcommand{\quark}{{\sc Carmot}\xspace}
\newcommand{\hlabel}[1]{\begin{hypothesis}\label{#1}\end{hypothesis}\ref{#1}}
\newcommand{\A}{\mathcal{A}}
\newcommand{\entails}{\vDash}
\newcommand{\M}{\mathcal{M}}
\newcommand{\colondash}{~\operatorname{:-}~}
\newcommand{\cedge}{\xrightarrow{c}}
\newcommand{\fedge}{\xrightarrow{f}}
\newcommand{\medge}{\xrightarrow{m}}
\newcommand{\fmedges}{\xrightarrow{f,m}\!\!^{*}}
\newcommand{\reaches}{\rightarrow^{*}}
\newcommand{\reachable}{\leftrightarrow^{*}}

% Low-level reduction relation
\newcommand{\rightarrowdbl}{\rightarrow\mathrel{\mkern-14mu}\rightarrow}

\newcommand{\xrightarrowdbl}[2][]{%
  \xrightarrow[#1]{#2}\mathrel{\mkern-14mu}\rightarrow
}
\newcommand{\qstepsto}{\xrightarrowdbl{\spc}}



%% Journal information
%% Supplied to authors by publisher for camera-ready submission;
%% use defaults for review submission.
%%\acmJournal{PACMPL}
%%\acmVolume{1}
%%\acmNumber{POPL} % CONF = POPL or ICFP or OOPSLA
%%\acmArticle{1}
%%\acmYear{2022}
%%\acmMonth{1}
%%\acmDOI{} % \acmDOI{10.1145/nnnnnnn.nnnnnnn}
%%\startPage{1}
\acmConference[PLDI]{ACM Conference}{June 2022}{San Diego, USA}%

%% Copyright information
%% Supplied to authors (based on authors' rights management selection;
%% see authors.acm.org) by publisher for camera-ready submission;
%% use 'none' for review submission.
\setcopyright{none}
%\setcopyright{acmcopyright}
%\setcopyright{acmlicensed}
%\setcopyright{rightsretained}
%\copyrightyear{2018}           %% If different from \acmYear

%% Bibliography style
\bibliographystyle{ACM-Reference-Format}
%% Citation style
%% Note: author/year citations are required for papers published as an
%% issue of PACMPL.
\citestyle{acmnumeric}   %% For author/year citations


%%%%%%%%%%%%%%%%%%%%%%%%%%%%%%%%%%%%%%%%%%%%%%%%%%%%%%%%%%%%%%%%%%%%%%
%% Note: Authors migrating a paper from PACMPL format to traditional
%% SIGPLAN proceedings format must update the '\documentclass' and
%% topmatter commands above; see 'acmart-sigplanproc-template.tex'.
%%%%%%%%%%%%%%%%%%%%%%%%%%%%%%%%%%%%%%%%%%%%%%%%%%%%%%%%%%%%%%%%%%%%%%


\begin{document}

%% Title information
\title[]{Runtime-Assisted Convergence in Replicated Data Types}
%% [Short Title] is optional;
                                        %% when present, will be used in
                                        %% header instead of Full Title.
%\titlenote{with title note}             %% \titlenote is optional;
                                        %% can be repeated if necessary;
                                        %% contents suppressed with 'anonymous'
%\subtitle{Subtitle}                     %% \subtitle is optional
%\subtitlenote{with subtitle note}       %% \subtitlenote is optional;
                                        %% can be repeated if necessary;
                                        %% contents suppressed with 'anonymous'


%% Author information
%% Contents and number of authors suppressed with 'anonymous'.
%% Each author should be introduced by \author, followed by
%% \authornote (optional), \orcid (optional), \affiliation, and
%% \email.
%% An author may have multiple affiliations and/or emails; repeat the
%% appropriate command.
%% Many elements are not rendered, but should be provided for metadata
%% extraction tools.

%% Author with single affiliation.
\author{Gowtham Kaki}
%\authornote{with author1 note}          %% \authornote is optional;
                                        %% can be repeated if necessary
%\orcid{nnnn-nnnn-nnnn-nnnn}             %% \orcid is optional
\affiliation{
%  \position{Position1}
%  \department{Department1}              %% \department is recommended
  \institution{University of Colorado Boulder}            %% \institution is required
%  \streetaddress{Street1 Address1}
%  \city{City1}
%  \state{State1}
%  \postcode{Post-Code1}
  \country{USA}                    %% \country is recommended
}
%\email{first1.last1@inst1.edu}          %% \email is recommended

%% Second Author
\author{Prasanth Prahladan}
%\authornote{with author2 note}          %% \authornote is optional;
                                        %% can be repeated if necessary
%\orcid{nnnn-nnnn-nnnn-nnnn}             %% \orcid is optional
\affiliation{
%  \position{Position2a}
%  \department{Department2a}             %% \department is recommended
  \institution{University of Colorado Boulder}           %% \institution is required
%  \streetaddress{Street2a Address2a}
%  \city{City2a}
%  \state{State2a}
%  \postcode{Post-Code2a}
  \country{USA}                   %% \country is recommended
}
%\email{first2.last2@inst2a.com}         %% \email is recommended

%% Third Author
\author{Nicholas Lewchenko}
%\authornote{with author2 note}          %% \authornote is optional;
                                        %% can be repeated if necessary
%\orcid{nnnn-nnnn-nnnn-nnnn}             %% \orcid is optional
\affiliation{
%  \position{Position2a}
%  \department{Department2a}             %% \department is recommended
  \institution{University of Colorado Boulder}           %% \institution is required
%  \streetaddress{Street2a Address2a}
%  \city{City2a}
%  \state{State2a}
%  \postcode{Post-Code2a}
  \country{USA}                   %% \country is recommended
}
%

%% Abstract
%% Note: \begin{abstract}...\end{abstract} environment must come
%% before \maketitle command
\begin{abstract}
  We propose a runtime-assisted approach to enforce convergence in
  distributed executions of replicated data types. The key
  distinguishing aspect of our approach is that it guarantees
  convergence \emph{unconditionally} -- without requiring data
  type operations to satisfy algebraic laws such as commutativity and
  idempotence. Consequently, programmers are no longer obligated to
  prove convergence on a per-type basis. Moreover, our approach lets
  sequential data types be reused in a distributed setting by
  extending their implementations rather than refactoring them.
  The novel component of our approach is a distributed runtime that
  orchestrates \emph{well-formed } executions that are guaranteed to
  converge. Despite the utilization of a runtime, our approach comes
  at no additional cost of latency and availability. Instead, we
  introduce a novel tradeoff against a metric called \emph{staleness},
  which roughly corresponds to the time taken for replicas to
  converge. We implement our approach in a system called \quark and
  conduct a thorough evaluation of its tradeoffs in the context of two
  case studies.
\end{abstract}


%% 2012 ACM Computing Classification System (CSS) concepts
%% Generate at 'http://dl.acm.org/ccs/ccs.cfm'.
\begin{CCSXML}
<ccs2012>
<concept>
<concept_id>10010147.10010919.10010177</concept_id>
<concept_desc>Computing methodologies~Distributed programming languages</concept_desc>
<concept_significance>500</concept_significance>
</concept>
<concept>
<concept_id>10010520.10010575.10010578</concept_id>
<concept_desc>Computer systems organization~Availability</concept_desc>
<concept_significance>500</concept_significance>
</concept>
<concept>
<concept_id>10011007.10011074.10011099.10011692</concept_id>
<concept_desc>Software and its engineering~Formal software verification</concept_desc>
<concept_significance>300</concept_significance>
</concept>
</ccs2012>
\end{CCSXML}

\ccsdesc[500]{Computing methodologies~Distributed programming languages}
\ccsdesc[500]{Computer systems organization~Availability}
\ccsdesc[300]{Software and its engineering~Formal software verification}
%% End of generated code


%% Keywords
%% comma separated list
\keywords{Replication, MRDT, CRDT, Runtime, Convergence, Collaborative
Editing}  %% \keywords are mandatory in final camera-ready submission


%% \maketitle
%% Note: \maketitle command must come after title commands, author
%% commands, abstract environment, Computing Classification System
%% environment and commands, and keywords command.
\maketitle


\section{Introduction}

Programming geo-distributed replicated systems is hard given the complexity
of reasoning about constantly evolving states at different replicas.
Uncoordinated updates to replicas lead to states that are unlikely to
converge. A prominent approach to overcome this problem is to reorganize
the replicated state in terms of Commutative Replicated Data Types (CRDTs).
CRDTs require operations on the replicated state to commute, thereby
allowing such operations to be executed in different orders at replicas.
Applications composed of CRDTs therefore needn’t make assumptions about the
behavior of the underlying network; as long as the network eventually
delivers all in-flight messages, the replica states are guaranteed to
eventually converge. This guarantee is called Strong Eventual Consistency
(SEC). SEC is a sufficient condition to ensure eventual convergence of
replicated state. As such it provides a principled approach to build
distributed applications by composing CRDTs. It is however worthwhile to
note that SEC is not (proven to be) a necessary condition, and therefore
does not rule out the possibility of alternative approaches towards
convergent replicated state. Such an alternative approach is particularly
attractive if it frees application programming from the constrains of
commutativity.

The current work (SC MRDTs) proposes one such approach. The key idea behind
SC MRDTs is to decouple the orthogonal concerns of distributed applications
that are conflated by CRDTs, namely (1). Availability: the ability to
immediately apply updates to local replica without coordination, and (2).
Convergence: the ability to eventually apply updates at remote replicas in
a way that guarantees convergence. Note that while the Availability concern
explicitly rules out inter-replica coordination, Convergence doesn’t. In
other words, it is permissible for replicas to coordinate among themselves
to achieve convergent replication, as long as such coordination doesn’t
make a replica unavailable to its users.  SC MRDTs exploit this
observation. Like CRDTs, SC MRDTs allow updates from user-submitted
operations to be applied immediately to the local replica to minimize the
user-perceived latency. Unlike CRDTs however, SC MRDTs do not allow updates
to be applied at remote replica at arbitrary time and in arbitrary order.
Instead, the system underlying SC MRDTs carefully orchestrates the
distributed execution so as to merge the competing states at different
replicas in an order that guarantees convergence. Thus, any data type with
a merge operation qualifies to be a replicated data type without having to
re-engineer its operations around the commutativity principle. 


\section{Motivation}
\label{sec:motivation}

In this section we motivate our runtime-assisted approach to
convergent RDTs with help of a set data structure.
% demonstrates the need for run-time intervention to ensure convergence.

\subsection{Add-Wins Set RDT}

Fig.~\ref{fig:ocaml-set} shows a simplified interface of a set data
type in OCaml. The interface hides a reference to a set (\C{Set.t}),
which can be updated in-place via \C{insert} and \C{remove}
operations. As such \C{Set} is an ordinary data type primarily
designed for sequential execution. Under a concurrent execution with
an asynchronously replicated state, \C{Set} would exhibit anamolous
behaviours such as those in Fig.~\ref{fig:crdt-execs}.

Fig.~\ref{fig:crdt-execs-1} shows an anamolous execution with three
replicas -- $R_1$, $R_2$, and $R_3$, all of which start with a
singleton set containing the element $e$. A client connects to the
replica $R_3$ and executes a \C{remove $e$} operation, which is then
asynchronously propagated to other replicas. Some time after applying
$R_3$'s \C{remove} at $R_2$, another client connects to $R_2$ and
re-adds $e$ by issuing an \C{insert $e$} operation. Consequently, the
state at $R_2$ is again the singleton set $\{e\}$. Replica $R_3$
however receives $R_2$'s \C{insert} ahead of $R_1$'s \C{remove},
applies them in the same order, and ends up with an empty set. The
execution thus results in divergent replica states.

Note that if set \C{insert} and \C{remove} commute, then executing
them in different order at $R_2$ and $R_3$ would not have led to
divergence. As such, \C{Set} is not a CRDT due to the admittance of
non-commutative operations. Nonetheless \C{Set} can be transformed
into a CRDT by re-engineering its interface and operations. For
instance, the anomalous execution in Fig.~\ref{fig:crdt-execs-1} can
be pre-empted by ensuring that updates are only ever applied in the
causal order.  This can be done by extending \C{Set} with vector
clocks to keep track of the causal history of each operation. A set
\C{insert} (resp. \C{remove}) would now generate an \C{Insert} (resp.
\C{Remove}) \emph{effect} tagged with the vector clock of the origin
replica. Here, the vector clock simply records the sequence number of
last operation from each replica whose effect has been received and
applied at the current replica. When an effect is received at a
replica, it is buffered until the time all its causally-preceding
effects (as captured by the tagged vector clock) have already been
received and applied. This strategy would preempt the execution in
Fig.~\ref{fig:ctrdt-execs-1} by buffering $R_2$'s \C{insert} at $R_1$
until the causally-preceding \C{remove} of $R_1$ is received and
applied. An interface for such a \emph{causally-consistent} set RDT is
shown in Fig.~\ref{fig:cc-set}. 

Unfortunately, \C{Set} data type of Fig.~\ref{fig:cc-set} still admits
divergent executions due to concurrent updates.
Fig.~\ref{fig:crdt-execs-2} describes one such execution. Here,
replicas $R_1$ and $R_2$ both start with a singleton set $\{e\}$.
Two distinct clients connect to $R_1$ and $R_2$ respectively and issue
two concurrent \C{remove $e$} operations. Later, another
client connects to $R_2$ and issues an \C{insert $e$} operation. The
effects of these operations are asynchronously applied at remote
replicas as shown in the figure, resulting in the divergent states at
$R_1$ and $R_2$.

Note that, unlike the execution in Fig.~\ref{fig:crdt-execs-1}, the
conflicting operations in Fig.~\ref{crdt-execs-2}, namely $R_1$'s
\C{remove} and $R_2$'s \C{insert}, are \emph{not} causally related,
hence their relative order is not determined by the sequential
specification of the data type. Forcing a causal relationship between
them requires synchronization between \C{insert}s and \C{remove}s,
which is expensive in an asynchronous distributed setting. It
therefore becomes inevitable to ascribe semantics to concurrent
executions to restore convergence. This is done by imposing an
\emph{arbitration order} among concurrent conflicting operations,
which are otherwise unordered. For example, in
Fig.~\ref{fig:crdt-execs-2}, we might let $R_2$'s \C{insert} override
$R_1$'s \C{remove} considering that \C{insert} is re-inserting an
element after a previous remove. Ordering concurrent removes ahead
of inserts uniformly on all replicas leads to an implementation of
\C{Set} RDT where concurrent (re-)insertions consistenty \emph{win}
over removals. Such an \emph{add-wins} set is useful, for instance, to
implement a shared shopping cart where two users can concurrently
remove an item from the cart, but if one of them re-adds it then the
item should be present in the final cart\footnote{This is in fact the
semantics of Amazon's shopping cart~\cite{dynamo}}.

Extending the \C{Set} implementation from Fig.~\ref{fig:cc-set} with
\emph{add-wins} semantics is however not trivial. The suggested
approach involves tracking element-wise causal dependencies between
the \C{remove}s and \C{insert}s by maintaining a vector clock
\emph{for each element} $e$ in the set~\cite{zawirski-thesis}. The
vector clock of $e$ records the sequence number of the last \C{insert
$e$} operation from each replica whose effect has been received and
applied at the current replica. The \C{Insert} and \C{Remove} effects
on $e$ will now be tagged with $e$'s vector clock. When an \C{Remove}
effect on $e$ is received at a replica, it is applied only if the
tagged vector clock is no less than $e$'s local vector clock, i.e.,
only if the arriving \C{Remove} has seen (i.e., causally succeeds) at
least those \C{insert $e$} operations the current replica is aware of.
Otherwise the effect is simply a no-op. This strategy would result in
convergent states despite the execution in Fig.~\ref{fig:crdt-execs-2}
as $R_2$'s \C{remove} effectively becomes a no-op at $R_1$ due to
there being at least one \C{insert} operation on $e$ that it hasn't
seen. The \C{Set} interface extended with element-wise vector clocks
is shown in Fig.~\ref{fig:rw-set}. The resultant set RDT is assumed to
be correct albeit a formal proof of convergence could not be found in
the literature. 

The above exercise demonstrates the considerable ingenuity and effort
involved in deriving a convergent RDT out of such simple data type as
\C{Set}. Some of the effort can be mitigated by strengthening the
underlying system model insofar as it doesn't affect the latency and
availability of the application. For instance, a system that always
delivers messages in the causal order would automatically preempt the
execution in Fig.~\ref{fig:crdt-execs-1} without the need for
additional intervention on behalf of the developer. This is a
particularly attractive proposition considering that causal
consistency can be ``bolted on'' an existing implementation of an
eventually consistent system without weakening its
guarantees~\cite{bolton}. Unfortunately, such strengthening of the
system model would deliver no benefits to the developer if they still
have to reason about fine-grained causal dependencies to guarantee
convergence, such as in the case of Fig.~\ref{fig:crdt-execs-2}. As we
observed earlier, the execution in Fig.~\ref{fig:crdt-execs-2} seems
inevitable unless every pair of \C{insert} and \C{remove} operations
are synchronized, which, regrettably, is not a practical option. The
developer therefore seems to be stuck.

Fortunately, there is a way out of this problem and the solution
requires a switch in the perspective of replication from an
operation-centric view characterized by explicit effects to a
state-centric view characterized by explicit state merges. The
motivation for this switch is the key observation that it is possible
to induce an ordering over \emph{states} of replicas even as their
operations and effects remain unordered, as long as such states are
\emph{mergeable}. The ordering is enforced by synchronizing merges in
the background while the replica-local execution progresses
unhindered. Consequently, eventual convergence can be guaranteed
\emph{without} impacting the user-perceived latency of the operations.
Moreover, working with states instead of operations frees the latter
from having to conform to algebraic laws such as commutativity.  Put
together, this means that the clients of a replicated data type can
perform any action allowed by the sequential version of the type and
immediately see the results of their actions on the local replica
while being safe in the knowledge that replica states will eventually
converge.

To demonstrate the aforementioned intuitions we reconsider the
executions from Fig.~\ref{fig:crdt-execs}, this time in the
state-centric replication model. The cornerstone of the state-centric
model is a three-way \C{merge} function that merges concurrent
versions of the state in presence of their (lowest) common ancestor
version. In our running example, the state is a value of type
\C{Set.t}, hence \C{Set.merge} function would have the type signature:
\begin{center}
\C{Set.merge} : \C{Set.t} $\rightarrow$ \C{Set.t} $\rightarrow$ \C{Set.t}
$\rightarrow$ \C{Set.t}
\end{center}
The three arguments of \C{merge} correspond to the lowest common
ancestor (LCA) version and the two concurrent versions that
independently evolved from the ancestor version. The LCA is a causal
ancestor of concurrent versions, hence causal consistency is built
into the replication model. The result of set merge, intuitively, must
contain the common elements in the two concurrent versions along with
any newly added elements in either versions. Concretely:
\begin{ocaml}
let merge s s1 s2 = 
              (s1 $\cap$ s2) $\cup$ (s1 - s) $\cup$ (s2 - s)
\end{ocaml}
Equipped with the above set merge function, we can now consider the
equivalent execution of Fig.~\ref{fig:crdt-execs-1} in state-centric
model. The execution is shown in Fig.~\ref{fig:mrdt-execs-1}. The
initial version on all three replicas ($v_0$) is the singleton set
$\{e\}$. Applying operations to replicas creates new versions, e.g.,
$v_1$ on $R_3$. Changes can be propagated by merging versions, e.g.,
version $v_2$ on $R_2$ is a result of merging $v_0$ and $v_1$ in
presence of their LCA $v_0$\footnote{Versions $v_0$ and $v_1$ are not
  concurrent as the former is an ancestor of the latter. Merging $v_1$
  into $v_0$ is nonetheless possible as $v_1$ is ahead of $v_0$ in
  causal order. In Git parlance this is a \emph{fast forward} merge.}.
Likewise $v_4$ on $R_1$ is created by merging $v_3$ and $v_0$ in
presence of their LCA $v_0$. By the end of the execution, versions $v_4$
and $v_3$ at $R_1$ and $R_2$ (resp.) have witnessed the same set of
operations, hence are in agreement.

Convergence however is not an inherent virtue of the state-centric
replication model. Fig.~\ref{fig:mrdt-exec-2} shows the state-centric
analogue of the execution in Fig.~\ref{fig:crdt-exec-2}, which too
diverges.


\section{Semantics of State-Centric Replication}
\label{sec:abstract-sem}

\begin{figure*}[t]
\begin{smathpar}
  \begin{array}{c}
    v\in\texttt{Versions/Vertices}\spc
    b\in\texttt{Branches}\spc
    c\in\texttt{CommitIds}\spc
    n\in\texttt{Values}\spc
    \cedge,\fedge,\medge \in \texttt{Edges}\spc
    G\in(\texttt{Vertices},\texttt{Edges})\\
    N : \texttt{Version} \rightarrow \texttt{Value}\spc
    C : \texttt{Version} \rightarrow \Pow{\texttt{CommitId}} \spc
    H : \texttt{Branch} \rightarrow \texttt{Version} \spc
    L : \texttt{Branch}\times\texttt{Branch} \rightarrow
    \texttt{Version}\\
  \end{array}
\end{smathpar}
%
\fbox {\( (G,N,C,H,L) \stepsto (G',N',C',H',L')\)} 
%
\bigskip

%
\begin{smathpar}
\begin{array}{c}
\RULE
{
  b\in dom(H)\spc
  v\not\in V\spc
  i\not\in codom(C)
}
{
  ((V,E),N,C,H,L) \stepsto (V \cup \{v\},\, E\cup\{H(b) \cedge v\},\,
  N[v \mapsto n],\,C[v \mapsto \{i\} \cup C(H(b))],\, H[b \mapsto v], L)
}
\spc
  [\rulelabel{Commit}]
\end{array}
\end{smathpar}
%

%
\begin{smathpar}
\begin{array}{c}
\RULE
{
  b\in dom(H)\spc
  b'\not\in dom(H) \spc
  v\not\in V\spc
}
{
  \hspace*{-0.3in}
  ((V,E),N,C,H,L) \stepsto (V \cup \{v\},\, E\cup\{H(b) \fedge v\},\,
  N[v \mapsto N(H(b))],\, C[v \mapsto C(H(b))],\\
  \hspace*{2in} 
  H[b' \mapsto v],\,
  L[(b',b) \mapsto H(b)]
                 [\{(b',b'') \mapsto L(b,b'') \,|\, b'' \neq b\}])
}
\spc
[\rulelabel{Fork}]
\end{array}
\end{smathpar}
%

%
\begin{smathpar}
\begin{array}{c}
\RULE
{
  b,b'\in dom(H)\spc
  C(H(b)) \supset C(L(b,b'))\spc
  C(H(b')) \supset C(L(b,b'))\\
% \neg(H(b') \reaches H(b))\spc
% \neg(H(b) \reaches H(b'))\\
  \forall (b''\in dom(H)).~L(b,b'') \reaches L(b',b'') 
    \disj L(b',b'') \reaches L(b,b'') \\
  n = {\sf merge}(N(L(b,b')),\, N(H(b)),\, N(H(b'))) \spc
  v \not\in V
}
{
  \hspace*{-1.5in}
  ((V,E),N,C,H,L) \stepsto (V \cup \{v\},\, E\cup\{H(b) \medge v, 
                                                 H(b') \medge v\},\\
    \hspace*{1in}
    N[v \mapsto n],\,
    C[v \mapsto C(H(b)) \cup C(H(b'))],\, H[b \mapsto v],\\
    \hspace*{1.8in}
    L[(b,b') \mapsto H(b') ]
     [\{(b,b'') \mapsto L(b',b'') \,|\, L(b,b'') \reaches L(b',b'')\}])
}
\spc
[\rulelabel{Merge}]
\end{array}
\end{smathpar}
%


%
\begin{smathpar}
\begin{array}{c}
\RULE
{
  b,b'\in dom(H)\spc
  C(H(b)) = C(L(b,b'))\spc
  C(H(b')) \supset C(L(b,b'))\\
% \neg(H(b') \reaches H(b))\spc
% H(b) \reaches H(b')\\
  \forall (b''\in dom(H)).~L(b,b'') \reaches L(b',b'') 
    \disj L(b',b'') \reaches L(b,b'') \spc
  v \not\in V
}
{
  \hspace*{-1.5in}
  ((V,E),N,C,H,L) \stepsto (V \cup \{v\},\, E\cup\{H(b) \medge v, 
                                                 H(b') \medge v\},\\
    \hspace*{1in}
    N[v \mapsto N(H(b'))],\,
    C[v \mapsto C(H(b'))],\,
    H[b \mapsto v],\\
    \hspace*{1.8in}
    L[(b,b') \mapsto H(b') ]
     [\{(b,b'') \mapsto L(b',b'') \,|\, L(b,b'') \reaches L(b',b'')\}])
}
\spc
[\rulelabel{FastFwd}]
\end{array}
\end{smathpar}
%

\caption{The semantics of \quark abstract machine inspired by the Git
version control system}
\label{fig:git-semantics}
\end{figure*}


In this section we present the formal semantics of our state-centric
replication scheme with linearized merges. As the informal development
from previous section suggests, our replication scheme is strongly
inspired by version control systems (VCS) such as Git.  We
embrace this analogy in our formal development to manifest distributed
executions over replicated state with help of an abstract ``Git''
machine building a well-formed version history graph.  We show that
the version history graph thus generated has several desirable
properties including unique LCAs for every pair of versions, and
convergence of versions that include the same set of \emph{commits}.
We are however not concerned about the practical aspects of the system
yet; the subsequent sections gradually refine the abstract semantics
described here into a practical distributed system we call \quark. For
convenience, we refer to the abstract ``Git'' machine we describe here
also as \quark.

Fig.~\ref{fig:git-semantics} shows the operational semantics of the
\quark abstract machine. The machine admits the usual version control
operations, namely \rulelabel{Commit}, \rulelabel{Fork},
\rulelabel{Merge}, and \rulelabel{FastFwd}. Operation
\rulelabel{FastFwd} is a special case of merge which simply \emph{fast
forwards} a branch to a later version. A \emph{branch} is a linear
sequence of versions, which intuitively denotes the progressive
evolution of the state on a replica. The latest version of the branch,
called its \emph{head}, denotes the current state of the corresponding
replica. A new head version is created either by an
externally-initiated \emph{commit} (\rulelabel{Commit}) or by merging
the current head with a concurrent or causally-succeeding version from
a different branch (\rulelabel{Merge} and \rulelabel{FastFwd}
respectively). A new branch can be created by \emph{forking off} a
version on an existing branch (\rulelabel{Fork}). Although a version
control system admits more operations (e.g., ``rebase''), we observe
that these four basic actions are sufficient to capture the behavior
of an asynchronously replicated multi-versioned state machine.

The state of \quark abstract machine is a tuple $\Delta = (G,N,C,H,L)$, where:
\begin{itemize}
  \item $G = (V,E)$ is the version history DAG generated by the
    execution of the abstract machine. Vertices ($V$) of the graph are
    the set of all versions that ever existed during the execution.
    Edges ($E$) record the relationships between various versions. In
    particular, there are three kinds of edges corresponding to the
    three basic operations: commit ($\cedge$), merge ($\medge$), fork
    ($\fedge$). Fast forward, being a special case of merge, is also
    denoted by a merge edge ($\medge$). Existence of an edge $v_0
    \rightarrow  v_1$ denotes that versions $v_0$ and $v_1$ are related by
    some operation. For e.g., $v_0 \fedge v_1$ denotes that $v_1$ is a
    new version forked-off from the version $v_0$ using the
    \rulelabel{Fork} rule. As usual, \emph{path} relation is the
    reflexive transitive closure of the edge relation, i.e,
    $\reaches$.
%   \rulelabel{Fork}
%   rule (explained later) dictates that $v_0$ and $v_1$ are different
%   versions belonging to different branches, albeit denoting the 

  \item $N : \texttt{Version} \rightarrow \texttt{Value}$ is a
    (partial) function (i.e., a map) that mas versions to their
    values.  We distinguish versions from values as different versions
    on different branches may store the same value (e.g., the same
    string ``hello''), yet need to be uniquely identified. The domain
    of values is left uninterpreted except for the requirement that
    it be \emph{mergeable}, i.e., define a three-way \C{merge}
    function.

  \item $C : \texttt{Version} \rightarrow \Pow{\texttt{CommitId}}$
    maps a version $v$ to the set of commit ids included in that
    version. Each commit event during the execution is uniquely
    identified by a commit id (analogous to the Git's commit hash).
    The set of commit ids $C(v)$ therefore denotes the commit events
    that contributed to the version $v$. Intuitively, this represents
    the set of user-initiated operations that affected the value of
    this version.

  \item $H : \texttt{Branch} \rightarrow \texttt{Version}$ is an
    injective (partial) function that maps each branch to its head
    version. 

  \item $L: \texttt{Branch}\times\texttt{Branch} \rightarrow
    \texttt{Version}$ maps a pair of branches to the lowest common
    ancestor (LCA) version of their heads. We later prove the unique
    LCA property, so $L$ is indeed a (partial) function. Note that LCA
    of a pair of branches $b_1$ and $b_2$ need not necessarily lie on
    $b_1$ or $b_2$; it could also be a version on a different branch
    $b$. This happens, for e.g., if $b_1$ and $b_2$ alternatively
    merged the same version from $b$.
\end{itemize}

\paragraph{Notation} We let $v_{\odot}$ denote the initial version
(``root''), and $b_{\odot}$ denote the initial branch (``master'').
The execution of the abstract machine progresses by adding to the sets
$V$ and $E$, and updating the maps $N$, $C$, $H$, and $L$. We adopt
the usual update notation, e.g., $H[b \mapsto v]$ is a map $H'$ such
that $H'(b) = v$, and forall $b' \neq b$, $H'(b') = H(b')$.  Multiple
updates to a map are parsed left-associatively. Map $L$ is assumed to
be commutative, so $L[(b,b') \mapsto v]$ is equal to $L[(b,b') \mapsto
v][(b',b) \mapsto v]$; the former is used as a succinct replacement of
the latter. To update multiple bindings in $L$, we use the set
comprehension notation: given a branch $b$, $L[\{(b,b') \mapsto v
\,|\, \phi(b,b')\}]$ updates all bindings $(b,b')$ in $L$ to $v$,
where $b'$ is any branch such that $\phi(b,b')$ is true. Domain and
co-domain of a map $M$ is denoted $dom(M)$ and $codom(M)$
respectively.

\begin{definition}[Initial State and Version History Graph]
  The graph $G_{\odot} = (\{v_{\odot}\},\emptyset)$ is the initial
  version graph. The state $\Delta_{\odot}$ = $(G_{\odot},\,
  [v_{\odot} \mapsto b_{\odot}],\, [b_{\odot} \mapsto v_{\odot}],\,
  \emptyset)$ is the initial state.
\end{definition}

\noindent All executions of the abstract machine are assumed to start from the
initial state.

\begin{definition}[Ancestor]
  \label{def:ancestor}
  In version history DAG $G = (V,E)$, version $v_0 \in V$ is said
  to be a (causal) ancestor of $v_1 \in V$ iff $v_0 \reaches v_1$.
  Versions $v_0, v_1 \in V$ are causally related iff either $v_0
  \reaches v_1$ or $v_1 \reaches v_0$.
\end{definition}

% \begin{definition}[Common Ancestor] Common ancestor is defined for
%   versions and branches as following:
%   \begin{itemize} 
%     \item In $G = (V,E)$, a version $v\in V$ is called a common ancestor of
%       versions $v_1 \in V$ and $v_2 \in V$ iff $v \reaches v_1$ and $v \reaches
%       v_2$ in $G$.
%     \item In state $\Delta \,=\, ((V,E),N,C,H,L)$, $v\in V$ is called
%       a common ancestor of branches $b_1 \in dom(H)$ and $b_2 \in
%       dom(H)$ iff $v$ is a common ancestor of $H(b_1)$ and $H(b_2)$.
%   \end{itemize}
% \end{definition}

\begin{definition}[Lowest Common Ancestor (LCA)]
  In version history DAG $G = (V,E)$, a version $v\in$ is a lowest common
  ancestor of versions $v_1 \in V$ and $v_2 \in V$ iff:
  \begin{itemize}
    \item $v$ is a common ancestor of $v_1$ and $v_2$, i.e., $v
      \reaches v_1$ and $v \reaches v_2$, and
    \item There does not exist a $v'\in V$ such that $v'$ is a common
      ancestor of $v_1$ and $v_2$, and $v \reaches v'$.
  \end{itemize}
  LCA of a pair of branches is defined as the LCA of their heads.
\end{definition}

\paragraph{Rules} The rule \rulelabel{Commit}
(Fig.~\ref{fig:git-semantics}) describes committing a new version $v$
onto the branch $b$ updating its head. The value $n$ for the new
version is assumed to have been provided by whoever has invoked the
commit, e.g., the user. Intuitively, user invokes an RDT operation
(e.g., \C{add($e$)}) on the value of the current version to create the
new value $n$, and thus the new version $v$. A unique commit id $c$ is
assigned to this commit event and added to $C(v)$. The set $C(v)$
also contains all the commit ids from the previous version $H(b)$
since the new value $n$ is assumed to have been derived from the
previous value $N(H(b))$. The edge $H(b) \cedge v$ records this
dependency and also documents the progression of branch $b$. The LCA
map $L$ does not change as the only new edge is between the versions
of the same branch $b$.

\rulelabel{Fork} describes the semantics of forking a new branch $b'$
from the head of an existing branch $b$. The head of $b'$ is a new
version $v$ that shares the same commit set ($C(v)$) and value
($N(v)$) as its predecessor ($H(b)$). The lowest common ancestor (LCA)
of $b'$ and its parent $b$ is clearly the head of the parent $H(b)$ as
there does not exist a version $v$ lower than $H(b)$ that is an
ancestor of both $H(b)$ and $H(b')$. For every other branch $b''$,
$L(b',b'')$ is same as $L(b,b'')$. Fork operation could model, for
e.g., creating a new replica by forking off the current state of an
existing replica.

\begin{figure}[ht]
  \centering
    \includegraphics[scale=0.4]{Figures/merge-precondition}
\caption{Explanation for the premise $L(b,b'') \reaches L(b',b'') \disj L(b',b'')
          \reaches L(b,b'')$ on the \rulelabel{Merge} rule.}
\label{fig:merge-precondition}
\end{figure}

\rulelabel{Merge} describes the semantics of merging the head of a
branch $b'$ into $b$ resulting in an new version $v$ on $b$.
Intuitively, \rulelabel{Merge} models the information exchange between
replicas. The two pre-conditions specified using the strict superset
relation ($\supset$) require each of the merging versions, $H(b)$ and
$H(b')$, to include at least one commit not present in their common
ancestor $L(b,b')$. These conditions ensure that the merge is not
trivial (trivial merge is handled by \rulelabel{FastFwd}). The next
pre-condition is key to ensuring the uniqueness of LCAs and the
linearity of merges. It requires that, for every branch $b''$ in the
system, the LCA of $b''$ with merging branches $b$ and $b'$ be
causally related, i.e, $L(b,b'') \reaches L(b',b'') \disj L(b',b'')
\reaches L(b,b'')$. Fig.~\ref{fig:merge-precondition} helps visualize
this condition in the most general case when (\rom{1}). Branches $b$,
$b'$, and $b''$ are distinct, and (\rom{2}). Their LCAs $L(b,b'')$ and
$L(b',b'')$ lie on a distinct pair of branches not equal to $b$, $b'$,
and $b''$. In the figure, once you merge $b'$ into $b$, every version
$v$ that is an ancestor of $L(b',b'')$, i.e., $v \reaches L(b',b'')$,
will be  a common ancestor of $H(b)$ and $H(b'')$. Clearly,
$L(b',b'')$ is the lowest among such common ancestors. But the current
lowest common ancestor of $b$ and $b''$ is $L(b,b'')$. We therefore
end up with two lowest common ancestors -- $L(b',b'')$ and $L(b,b'')$,
\emph{unless} both are ancestrally related. Thus the pre-condition
$L(b,b'') \reaches L(b',b'') \disj L(b',b'') \reaches L(b,b'')$. If
$L(b',b'') \reaches L(b,b'')$, then $L(b,b'')$, the current LCA of $b$
and $b''$, is still the LCA after the merge. On the other hand, if
$L(b,b'') \reaches L(b',b'')$, then $L(b',b'')$ becomes the lowest
common ancestor of $b$ and $b''$ after the merge. Thus $L(b,b'')$
needs to be updated if and only if $L(b,b'') \reaches L(b',b'')$. The
conclusion of the \rulelabel{Merge} captures this update using the set
comprehension notation. It also updates the LCA of the merging
branches $L(b,b')$ to $H(b')$ since the head of $b'$ is merged into
$b$. Updates to the other components of the state (e.g., $C$) follow
the same rationale as previous rules. Two edges are added to $E$ as
the new version $v$ is a descendant of the two merging versions.

Another notable aspect of the \rulelabel{Merge} rule is the invocation
of the \C{merge} function on the values of the  merging versions and
their common ancestor to derive the result of the merge. As explained
above, our development is parameterized on the domain of values for
which a \C{merge} function is defined. No further constraints are
imposed on \C{merge}. We use an uncurried version of \C{merge} to
avoid clutter.

\rulelabel{FastFwd} generalizes merge to the case when the merging
version $H(b')$ is a descendant of the version $H(b)$. In this case
$H'(b)$, the new head of $b$, needs to have the same value and same
set of commits as $H(b')$. Other premises and conclusions are similar
to the \rulelabel{Merge} rule. Note that we don't need a separate rule
for fast forward merge if \C{merge} satisfies the invariant that
$\forall n, n'.~ \C{merge}(n,n,n') \,=\, \C{merge}(n,n',n) \,=\, n$.
Having a separate rule lets us elide this constraint.

\paragraph{Properties} We now formalize the notable properties of the
abstract machine and its executions\footnote{
  The manual proofs of the theorems and their Ivy
  formalization~\cite{ivy} can be found in the supplementary material.
}. 

\begin{lemma}[Uniqueness of LCA]
  \label{lem:lca-uniqueness}
  In every reachable state $\Delta = (G,N,C,H,L)$ of the abstract
  machine, every pair of branches $b_1, b_2 \in dom(H)$ has a unique
  LCA given by $L(b_1,b_2)$.
\end{lemma}

The intuition behind the proof is succinctly captured by
Fig.~\ref{fig:merge-precondition}, which is explained above.

\begin{lemma}[Commit sets grow monotonically]
  \label{lem:commit-monotonicity}
  In every reachable state $\Delta = ((V,E),N,C,H,L)$ of the abstract
  machine: Forall $v_1,v_2 \in V$, if $v_1 \reaches v_2$ then $C(v_1)
  \subseteq C(v_2)$.
\end{lemma}

Lemma~\ref{lem:commit-monotonicity} guarantees that merges never lose
a commit.

\begin{corollary}[Commit sets modulo LCA are disjoint]
  \label{lem:commit-disjointness}
  In every reachable state $\Delta = ((V,E),N,C,H,L)$ of the abstract
  machine: Forall distinct $b_1, b_2 \in dom(H)$, and $v_0, v_1, v_2
  \in V$ s.t.  $v_1 = H(b_1)$ and $v_2 = H(b_2)$ and $v_0 =
  L(b_1,b_2)$, the following is true: $(C(v_2) - C(v_0)) ~\cap~
  (C(v_1) - C(v_0)) = \emptyset$.
\end{corollary}

Corollary~\ref{lem:commit-disjointness} follows from
Lemmas~\ref{lem:lca-uniqueness} and~\ref{lem:commit-monotonicity}.

\begin{theorem}[{\bf Convergence}]
  \label{thm:convergence}
  In every reachable state $\Delta$ = $((V,E),N,C,H,L)$ of the abstract
  machine: Forall distinct $b_1, b_2 \in dom(H)$, and $v_1, v_2 \in V$
  such that $v_1 = H(b_1)$ and $v_2 = H(b_2)$, the following is true:
  $C(v_1) = C(v_2) \Rightarrow N(v_1) = N(v_2)$.
\end{theorem}

Theorem~\ref{thm:convergence} is the key result of this section. It
asserts that any two branches that witnessed the same set of commits
have the same value. Intuitively, this means that any two replicas
that witnessed the same set of user actions arrive at the same final
state \emph{regardless} of the order in which they are witnessed. 

Convergence is vacuously true if the abstract machine never lets any
merges to happen, i.e., if the premises of the \rulelabel{Merge} rule
are too strong to be never true. We prove that this is not the case
with help of the following theorem:

\begin{theorem}[{\bf Progress}]
  \label{thm:progress}
  Every reachable state $\Delta = ((V,E),N,C,H,L)$ of the abstract
  machine is either: 
  \begin{itemize}
    \item A ``quiescent'' state, where: $\forall b_1,b_2 \in
      dom(H).~C(H(b_1)) = C(H(b_2))$, Or
    \item An ``unstuck'' state, where there exist $b_1,b_2 \in dom(H)$
      that satisfy the pre-conditions of \rulelabel{Merge} or
      \rulelabel{FastFwd} rules, i.e., $b_1$ and $b_2$ are mergeable. 
  \end{itemize}
\end{theorem}

Convergence and Progress together ensure the soundness of the \quark
abstract machine.


\section{Concrete Semantics}
\label{sec:concrete-sem}

\begin{figure*}[t]
\begin{smathpar}
  \begin{array}{c}
    i,j\in\texttt{ReplicaIds}\spc
    b_i,b_j\in\texttt{Branches}\spc
    t\in\texttt{VersionVectors}\spc
    n\in\texttt{Values}\spc
%   \cedge,\fedge,\medge \in \texttt{Edges}\spc
%   G\in(\texttt{Vertices},\texttt{Edges})\\
    N : \texttt{VersionVector} \rightarrow \texttt{Value}\\
%   C : \texttt{Version} \rightarrow \Pow{\texttt{CommitId}} \spc
%   R : \texttt{ReplicaId} \rightarrow \texttt{Branch} \spc
    H : \texttt{ReplicaId} \rightarrow \texttt{Branch} \rightarrow \texttt{VersionVector} \spc
%   L : \texttt{Branch}\times\texttt{Branch} \rightarrow \texttt{Version}\\
    B : \texttt{ReplicaId} \rightarrow (\texttt{Branch} \times \mathbb{N}) \rightarrow \texttt{VersionVector} \spc
  \end{array}
\end{smathpar}
%
  \fbox {\( (B,N,H) \qstepsto (B',N',H')\)} 
%
%\bigskip

%
\begin{smathpar}
\begin{array}{c}
\RULE
{
  b_i\in dom(H_i)\spc
  t = H_i(b_i)\spc
  t' = t[b_i \mapsto t(b_i) + 1]\spc
}
{
  (B,N,H) \qstepsto (B_i[(b_i,t'(b_i)) \mapsto t'],\,
  N[t' \mapsto n],\, H_i[b_i \mapsto t'])
}
\spc
  [\rulelabel{Commit}]
\end{array}
\end{smathpar}
%

%
% \begin{smathpar}
% \begin{array}{c}
% \RULE
% {
%   b_i\in dom(H)\spc
%   b_j\not\in dom(H) \spc
%   t = H_i(b_i)\spc
%   t' = t[b_j \mapsto 1]
% }
% {
%   (B,N,H) \qstepsto (B_j[(b_j,1) \mapsto t'],\, N[t' \mapsto N(t)],\,
%         H_j[b_j \mapsto t'])
% }
% \spc
% [\rulelabel{Fork}]
% \end{array}
% \end{smathpar}
%

%
\begin{smathpar}
\begin{array}{c}
\RULE
{
  b_i,b_j\in dom(H_i)\spc
  \forall (k\in dom(H)).~b_k \in dom(H_i) \conj H_i(b_k) = H_k(b_k)\\
  t_i = H_i(b_i) \spc
  t_j = H_i(b_j)\spc
  t_l = t_i \sqcap t_j\spc
  t_l < t_i\spc
  t_l < t_j\spc
  t' = t_j \sqcup t_i[b_i \mapsto t_i(b_i)+1]\\
  \forall (k\in dom(H)).~
    (t_i \sqcap H_i(b_k)) \lesseqgtr (t_j \sqcap H_i(b_k))\spc
  n = {\sf merge}(N(t_l),\, N(H_i(b_i)),\, N(H_i(b_j))) 
}
{
  (B,N,H) \qstepsto (B_i[(b_i,t'(b_i)) \mapsto t'],
            N[t' \mapsto n],\, H_i[b_i \mapsto t'])
}
\spc
[\rulelabel{Merge}]
\end{array}
\end{smathpar}
%


%
% \begin{smathpar}
% \begin{array}{c}
% \RULE
% {
%   b_i,b_j\in dom(H_i)\spc
%   \forall (k\in dom(H)).~b_k \in dom(H_i) \conj H_i(b_k) = H_k(b_k)\\
%   t_i = H_i(b_i) \spc
%   t_j = H_i(b_j)\spc
%   t_i < t_j \spc
%   t' = t_j[b_i \mapsto t_i(b_i)+1]\spc
%   \forall (k\in dom(H)).~
%     (t_i \sqcap H_i(b_k)) \lesseqgtr (t_j \sqcap H_i(b_k))
%   %
% % \neg(H(b') \reaches H(b))\spc
% % H(b) \reaches H(b')\\
% }
% {
%   (B,N,H) \qstepsto (B_i[(b_i,t'(b_i)) \mapsto t'],
%             N[t' \mapsto N(t_j)],\, H_i[b_i \mapsto t'])
% }
% \spc
% [\rulelabel{FastFwd}]
% \end{array}
% \end{smathpar}
%



%
\begin{smathpar}
\begin{array}{c}
\RULE
{
  b_k\in dom(H_i)\spc
  b_k \in dom(H_j)\spc
  t = H_i(b_k) \spc
  t' = H_j(b_k)\spc
  t' > t
  %
% \neg(H(b') \reaches H(b))\spc
% H(b) \reaches H(b')\\
}
{
  (B,N,H) \qstepsto (B_i[(b_k,t(b_k)+n) \mapsto B_j(b_k,t(b_k)+n)
                          \,|\, n \in \{1,\ldots,t'(b_k) - t(b_k)\}],
            N,\, H_i[b_k \mapsto t'])
}
\spc
[\rulelabel{Sync}]
\end{array}
\end{smathpar}
%



\caption{The semantics of \quark distributed machine}
\label{fig:quark-semantics}
\end{figure*}


The \quark abstract machine of the previous section deliberately
ignores system-level concerns to focus on the semantics of
version-controlled state replication. In this section we present the
\quark \emph{distributed} machine that addresses the key system-level
concerns and provides a blueprint for a practical version-controlled
replicated state machine. We first reify the the one-to-one
correspondence between replicas and branches we assumed informally in
the previous section. This requires us to relax the assumption of the
state $\Delta$ being shared synchronously across all replicas. Next we
present an efficient method to track the version history with help
of vector clocks. Finally, we outline an algorithm for garbage
collecting older versions so that the version history need not grow
unboundedly.

Fig.~\ref{fig:quark-semantics} shows the operational semantics of the
distributed machine. The key difference from the abstract semantics of
Fig.~\ref{fig:git-semantics} is the presence of $\texttt{ReplicaId}$
indexing the components of the system state. We thus admit the
possibility of different replicas having different conceptions of the
state. Another major difference is the use of version
vectors~\cite{vectorclock} as identifiers and placeholders for
versions. A version vector of a version $v$ records the sequence
number of the last version from each branch that causally precedes $v$
(as per Def.~\ref{def:ancestor}).  Concretely, version vector is a map
from branches to natural numbers:
\begin{center}
  $\texttt{VersionVector}$ = $\texttt{Branch} \rightarrow \mathbb{N}$
\end{center}

The state of the distributed machine is the triple $\delta = (B,N,H)$.
Components $B$ and $H$ are indexed by $\texttt{ReplicaId}$ to let us
denote their replica-local copies. For instance, replica $i$'s copy of
the \emph{head map} $H$ is given by $H\;i$, which we abbreviate to
$H_i$ for notational convenience. Like $H$ in
Fig.~\ref{fig:git-semantics}, $H_i$ maps each branch to the version
vector of its head. Note that two replicas may be out of sync w.r.t
the information in $H$ and $B$.  For instance, $H_i(b)$ may not be
equal to $H_j(b)$ if replicas $i$ and $j$ are out of sync. On a
replica $i$, \emph{branch map} $B_i$ gives the version vector
corresponding to a particular sequence number on a branch. For
instance, the version vector of the first version on branch $b$ is
given by $B_i(b,1)$ on replica $i$. The value map $N$ maps version
vectors to their values. We assume a single copy of $N$ to simplify
the presentation. Generalizing Fig.~\ref{fig:quark-semantics} to allow
for replica-local copies of $N$ is straightforward.

Conspicuous by its absence in Fig.~\ref{fig:quark-semantics} is the
LCA map $L$, which we previously used to track the LCA version for
every pair of branches. The use of version vectors, coupled with the
unique LCA guarantee, obviates the need for an LCA map.  We can
instead identify the LCA of a pair of versions $v_1$ and $v_2$ by
computing the \emph{greatest lower bound} (GLB) of their version
vectors $t_1$ and $t_2$. Concretely:
\begin{center}
  $t_l \,=\, t_1 \sqcap t_2$
\end{center}
Where $t_l$ is the version vector of the LCA and $\sqcap$ is the GLB
operator. Conversely, the version vector indentifying the result of a
merge can be computed as the \emph{least upper bound} (LUB) of the two
version vectors involved in the merge. Concretely:
\begin{center}
  $t_m \,=\, t_1 \sqcup t_2$
\end{center}
Where $t_1$ and $t_2$, and $t_m$ are the version vectors of the
merging versions and the result of the merge, respectively. For
technical reasons, however, we increment the component of $t_m$
corresponding to the current branch $b$ to signify that this is a new
version on $b$. So the actual version vector of the merge is $t_m' =
t_m[b \mapsto t_m(b)+1]$. The GLB and LUB operations on version
vectors are standard~\cite{vectorclock} -- GLB is computed by taking
the component-wise minumum, and LUB by taking their maximum. The
comparison of version vectors is also standard -- $t_1 < t_2$ iff
$\forall b.~t_1(b) < t_2(b)$. Clearly, not all version vectors are
comparable. We write $t_1 \lesseqgtr t_2$ if vectors $t_1$ and $t_2$
are comparable.

\paragraph{Notation and Conventions} We enforce a one-to-one mapping
between replicas and branches by adopting the convention that branch
$b_i$ always corresponds to replica $i$. Concretely this means that
replica $i$ only ever creates new versions on branch $b_i$. Also,
replica $i$ can only update its local copies of $H$ and $B$, i.e.,
$H_i$ and $B_i$. For e.g., $H_i[b_i \mapsto t]$ updates the head of
branch $b_i$ to $t$ on replica $i$.  Since $H_i$ is simply an
abbreviation of $H\;i$ in the formalism, $H_i[b_i \mapsto t]$ actually
expands to $H_i[i \mapsto b_i \mapsto t]$. We exploit this notation in
Fig.~\ref{fig:quark-semantics}. 
% When two replicas, $i$ and $j$, are involved in an update, e.g.,
% \rulelabel{Fork}, we write $H_{\tuplee{i,j}}[b_j \mapsto t]$ to mean
% $H[i \mapsto b_j \mapsto t][j \mapsto b_j \mapsto t]$.

\paragraph{Rules} The transition relation $\qstepsto$ of the \quark
distributed machine is defined in Fig.~\ref{fig:quark-semantics}.
There is one rule corresponding to every rule in
Fig.~\ref{fig:git-semantics} (abstract semantics). \rulelabel{Commit}
describes replica $i$ committing a new version on the branch $b_i$
with the given value $n$.  The vector $t'$ for the new version is
obtained from that of the previous version $t$ by incrementing the
component $b_i$. The sequence number of the new version on $b_i$ is
$t'(b_i)$ and the branch map $B_i$ is updated to reflect that. Note
that for all $j \neq i$, $H_j$ and $B_j$ remain unchanged indicating
that other replicas are not (yet) aware of this commit.
%
\rulelabel{Fork} forks off a new branch $b_j$ from $b_i$. A new
replica $j$ is assumed to take over $b_j$. The version vector $t'$ of
the branch head now has a new component $b_j$ mapped to $1$ to denote
this is the first version on $b_j$. Map $B_j$ is updated accordingly.
Value $N(t')$ is same as its parent $N(t)$.

\rulelabel{Merge} describes the semantics of replica $i$ merging a
concurrent version from a (remote) replica $j$. The merging versions
are heads of their respective branches $b_i$ and $b_j$.  Like its
counterpart in Fig.~\ref{fig:git-semantics}, \rulelabel{Merge} insists
that the LCAs of every other branch $b_k$ with $b_i$ and $b_j$ be
causally related. This condition is however expressed in terms of
version vectors with help of the GLB ($\sqcap$) and comparision
($\lesseqgtr$) operators. For the LCA determination to be sound,
the merging replica $i$ needs to have an accurate conception of the
current version history. Futhermore, there cannot be a concurrent
merge happening elsewhere that undermines the judgment of replica $i$
(reg. the safety of $i \leftarrow j$ merge). These requirements are
enforced by the premise $\forall (k\in dom(H)).~b_k \in dom(H_i) \conj
H_i(b_k) = H_k(b_k)$, which insists that replica $i$'s knowledge of
every other branch $k$ be current. This condition effectively
linearizes merges by requiring a merge to either \emph{see} or
\emph{be seen} by every other merge. In practice this is achieved
through global coordination (Sec.~\ref{sec:implementation}). Note that
\rulelabel{Merge} only pre-empts a concurrent \rulelabel{Merge}, not a
concurrent \rulelabel{Commit}. A remote replica $k$ is allowed to keep
committing new versions on to $b_k$ even as it remains unaware of the
merge on replica $i$. Such leniency is imperative if the system were
to retain the performance benefits of asynchronous replication.
\rulelabel{FastFwd} is similar to \rulelabel{Merge}, so we don't
discuss it separately.

The rule \rulelabel{Sync} captures asynchronous communication between
a pair of replicas ($i$ and $j$) to get one of them ($i$) up-to-date
with the other ($j$).  Unlike the other rules, \rulelabel{Sync} does
not extend the version history graph, and therefore has no counterpart
in Fig.~\ref{fig:git-semantics}. It merely updates replica $i$'s
knowledge of branch $b_k$ if replica $j$ happens to have later updates
from $k$, i.e., $H_j(b_k)$ happens to be ahead of $H_i(b_k)$. The new
versions on $b_k$ known to $j$ but not $i$ are then simply transferred
to $i$.




\section{Implementation}
\label{sec:implementation}

We realize a prototype of \quark runtime for MRDTs as a lightweight
shim layer on top of Scylla -- an off-the-shelf distributed data
store~\cite{scylla}. We rely on Scylla for inter-replica
communication, data replication, persistence, and fault tolerance.
\quark translates the high-level MRDT implementations in OCaml to
their low-level representations in the backing store and orchestrates
their well-formed distributed executions.
Fig.~\ref{fig:implementation} illustrates the overall architecture.

The implementation of \quark largely follows the design of \quark
distributed machine described in Sec.~\ref{sec:concrete-sem}.  We
manifest each component of the state, namely the branch map $B$, the
value map $N$, and the head map $H$, as a column family (i.e., a
table) in Scylla. The synchronization needed to linearize merges is
implemented with help of Scylla's support for conditional updates (CAS
operations) and expiring columns. The total order among merges is
enforced with help of \C{Quorum} reads and writes. Each user process
is assigned its own branch containing a replica of the MRDT.  Version
vectors are realized as associative lists and stored in Scylla as
blobs.

\begin{figure}[ht]
  \centering
    \includegraphics[scale=0.35]{Figures/implementation2}
\caption{\quark implementation architecture}
\label{fig:implementation}
  \vspace*{-0.2in}
\end{figure}

The implementation however differs from the formalization in two
significant ways. First is in the treatment of MRDT values.
Formalization assumes values to be atomic with no sharing in between
them. In practice, however, an MRDT could be a linked data structure
such as a binary tree, and two such values could share a significant
amount of internal structure.  Consequently, the size of the
\emph{diff} between two consecutive versions of a value could be
asymptotically less then the size of the data structure itself, in
which case it unreasonable to transfer the entire data structure over
the network. To facilitate the efficient computation of diffs between
versions of data structures, we implement a \emph{content-addressable
store} as a key-value table in Scylla where key is simply the SHA256
hash of the value. A linked data structure is stored as a collection
of nodes, where each node links to the other by referring to its hash.
The diff between two consecutive versions of a data structure would
simply manifest as new entries in the content-addressable store
reachable from the root of the new version. The new entries, being new
data, are automatically replicated by Scylla, thus letting us
reconstruct the new version at a remote location. The root hash of the
new version is obtained from the value map $N$, which now maps version
vectors to the \emph{hashes} of values.

The second significant difference between the implementation and the
formalization is the timing of merges. Operational Semantics in
Sec.~\ref{sec:abstract-sem} and Sec.~\ref{sec:concrete-sem} interleave
commits and merges such that only one of them executes at any given
time. Since commits are initiated by the user in practice,
interleaving them with strongly consistent merges increases their
latency as perceived by the user. For the user-perceived latency to
remain unaffected, it is important that a replica be always available
to execute user requests. \quark ensures this by handling user
requests in a foreground thread that is always allowed to commit to
the local branch. Merges, on the other hand, are relegated to the
background. A background thread constantly scans the remote branches
for new versions, and if there are any, merges them into the latest
version on the local branch after checking the necessary preconditions
(Sec.~\ref{sec:abstract-sem}). 

\quark's background merges however pose a new problem as they create
new versions on the local branch in the background while a user
operation is manipulating an older version in the foreground. When
the user attempts to write their version to the store, simply
committing it would effectively override the concurrent updates from
other users obtained via background merges. The solution, fortunately,
is straightforward: \quark \emph{merges} the user-submitted version
with the latest version on the local branch to create a new version
that includes the updates from either direction. Since this merge is
fully confined to the local replica, it is guaranteed to not affect
the LCA of the local branch in relation to any remote branch. To the
external world, it appears as if the local branch has simply committed
a new version that was derived from the older version. Since the
commit operation now entails a merge, the merged version has to be
returned to the user as the result of the commit.  Consequently the
\C{write} operation exposed by \quark is a function of type
$\texttt{Value} \rightarrow \texttt{Value}$, where $\texttt{Value}$ is
any MRDT.


\section{Evaluation}
\label{sec:eval}

In this section we present an emperical evaluation of \quark on two
case studies. First is a collaborative document editing
application -- a common usecase addressed by several CRDT
proposals~\cite{rga, treedoc, crdts}. Second is a replicated key-value
store implemented using a mergeable Red-Black tree data structure.

\subsection{Collaborative Editing}

\quark's MRDT approach obviates the need to build a dedicated
replicated data type for collaboratively-edited documents; an ordinary
document format extended with a merge operation would suffice. While
many data structures exist to represent text documents (e.g.,
ropes~\cite{boehm95}), we decided to adopt the simplest representation
of a document as a list of characters. 
\begin{center}
\begin{ocaml}
        type doc = char list
\end{ocaml}
\end{center}
While being simple, the advantage of this presentation is that we can
simply reuse the three-way \C{List.merge} function of the list data
type to merge documents. \C{List.merge} is a simple implementation of
list merge algorithm (in ~60 lines of OCaml) inspired by the GNU
\C{diff3} algorithm~\cite{gnudiff}. We thus adopt a straightforward
approach to building a collaborative document editor with the
intention to keep the development effort low enough to be easily
replicated. The convergence guarantee of \quark ensures that the
simplicity of our implementation doesn't come at the expense of
correctness. The aim of the experimental evaluation is to quantify the
impact of \quark on the performance.

Our experiment setup consists of multiple collaborators simultaneously
editing a 10000+ line document obtained from the Canterbury
Corpus~\cite{canterbury}. Each user holds a replica of the document
and is assumed to be editing the document at the speed of 240
characters per minute or 1 character every 0.25s. At 6 characters per
word, this amounts to 40 words per minute, which is the average typing
speed of humans. Each edit is immediately persisted to the disk by
creating a new version in the backing store. Thus there are at least
as many versions of the document as there are edits. Such extensive
versioning may be considered excessive in practice and could be
disabled. Each user process runs a \quark thread that commits
user-generated document versions to the local branch, while
merging with the concurrent versions from the remote branches in the
background. Each merge is synchronized as described in previous
sections.

\quark's background merges however pose a new problem as they create
new versions on the local branch in the background while the user is
busy editing an older version. When the user attempts to write their
version of the document to the store, simply committing it would
effectively override the concurrent updates from other users obtained
via background merges. The solution, fortunately, is straightforward:
we \emph{merge} the user-submitted value with the (value of) the
latest version on the local branch to create a new version that
includes the updates from either direction. Since this merge is fully
confined to the local replica, it is guaranteed to not affect the LCA
of the local branch with respect to a remote branch. To the external
world, it appears as if the local branch has simply committed a new
version.
% The LCA for this merge is
% simply the last version on the local branch read by the user. This
% merge need not be synchronized as it doesn't alter the LCAs between
% any two branches. The merged value can then returned to the user as
% the result of the write.
Thus, \quark's \C{write} is a function of type $\texttt{Value}
\rightarrow \texttt{Value}$, where $\texttt{Value}$ is \C{doc} in the
current application.

\noindent\paragraph{Latency} To measure the impact of \quark runtime on user
writes, we measure the latency of the \C{write} operation, which
includes the time spent merging the user version with the current
version, and persisting the resultant version to the store. We conduct
the experiments on a three-node cluster of \C{i3.large} machines in
Amazon \C{us-west2} data center. Each user connects to one of the
machines, forks a new branch, and performs 1000 edits in succession,
saving the document after each edit. We progressively increase the
number of concurrent users editing the document from 3 to 200 and
measure the impact of the increased concurrency on write latency.
Fig.~\ref{fig:latency} shows the 10th, 50th (median), and 90th
percentile latency values. The median and 10th percentile latencies
remain more-or-less constant with a slight increase between 3 and 60
concurrent users. This demonstrates that \quark has negligible effect
on the per-operation latency, which is expected. The 90th percentile
latency, however, shows no clear pattern by the virtue of being more
susceptible to transient system conditions. Nonetheless, the maximum
value of 90th percentile latency measured (32ms) remains well below
the the time between consecutive edits (0.25s), making it hard to
perceive by a human user. 

\begin{figure}[ht]
  \centering
    \includegraphics[scale=0.4]{Figures/monkey_latency}
\caption{Latency of writes to a shared document under \quark. }
\label{fig:latency}
\end{figure}


For comparison against a baseline, we have implemented an ``SC''
approach which achieves convergence by synchronizing each operation,
i.e., executing it under strong consistency (SC). The SC
implementation shares most of its code with \quark with the only
change being that it wraps commits instead of merges inside a lock.
As with \quark, we measured the write latency of the SC implementation
while increasing the number of concurrent users ($n$).  The median SC
write latency increases super-linearly from 10ms for $n=3$ to 63s
(i.e., >1m) for $n=60$, which is considerably more than the inter-edit
latency of 0.25s.


\noindent\paragraph{Staleness} Our implementation of \quark relies on
Scylla to replicate the contents of each branch across all the
replicas as fast as the network allows. However, for a user $A$ to see
the changes made by the other user $B$, the changes have to be
reflected in $A$'s local version, which can only happen through a
merge operation. Since \quark synchronizes merge operations globally,
it induces additional delay before $A$ can see $B$'s changes. We call
this additional delay \emph{staleness} as with the progression of
time, $B$'s version known to $A$ becomes increasingly stale. At the
system-level, an increase in staleness effectively delays the
convergence (but doesn't preempt it, as proved by
Theorem~\ref{thm:progress}). 

\begin{figure}[ht]
  \centering
    \includegraphics[scale=0.4]{Figures/monkey_staleness}
\caption{Staleness increases as the number of concurrent editors
  increase.}
\label{fig:monkey-staleness}
  \vspace*{-0.2in}
\end{figure}

To understand the effect of \quark on
staleness, we quantify and measure it along with latency in the
experiment setup described above. Staleness is defined as the time
taken for a version committed on one replica to be merged into a
concurrent version on a remote replica. To measure staleness, we
annotate every version $v$ with the timestamp $t$ of capturing the
wall clock time of its creation. When $v$ is is merged into a remote
branch $b$ at a later time $t'$, the difference $t' - t$ denotes the
staleness of $v$ w.r.t the new version on $b$. One such staleness
measurement is recorded for every merge that ever happens during the
experiment. We compute 10th, 50th, and 90th percentiles of staleness
values thus obtained. Fig.~\ref{fig:monkey-staleness} shows the
results. As evident, staleness increases steadily with the increasing
number of concurrent replicas, which is expected considering that
merges are synchronized, and increasing the number of replicas reduces
the number of opportunities for a replica to merge. The increase is
roughly linear in the number of replicas as our implementation passes
the lock around in a round-robin fashion, thus increasing the lock
latency in proportion to the number of replicas. Nonetheless, the 90th
percentile staleness values remain low -- in the order of 100ms with
the number of replicas under 30, and in the order of 1s with number of
replicas under 60. While further optimizations might reduce staleness,
a non-trivial staleness overhead is inevitable in \quark due to our
use of synchronized merges to guarantee convergence. 

% Our approach thus
% introduces staleness as novel tradeoff against convergence. While
% trading off staleness may not be an optimal choice for every
% application, it is arguably a less disruptive choice if an application
% has to choose between latency and staleness.
% Sec.~\ref{sec:motivation}\footnote{
%   Git admits anamolous version history graphs where two branches can
%   have the same set of commits and yet differ in their final version.
%   Supplementary material describes two such cases we observed on
%   Github.}. 

\subsection{Key-Value Store}

We repeat our experiments on a mergeable key-value store application
we implemented using a Red-Black tree MRDT. Latency results are shown
in Fig.~\ref{fig:rb-latency}, and staleness results in
Fig.~\ref{fig:rb-staleness}.

\begin{figure}[ht]
  \centering
    \includegraphics[scale=0.4]{Figures/rbmonkey_latency}
  \caption{Latency for Key-Value store operations under \quark}
\label{fig:rb-latency}
  \vspace*{-0.2in}
\end{figure}

\begin{figure}[ht]
  \centering
    \includegraphics[scale=0.4]{Figures/rbmonkey_staleness}
  \caption{Key-Value store staleness under \quark}
\label{fig:rb-staleness}
  \vspace*{-0.2in}
\end{figure}

Our experiments bring to the fore an inherent tradeoff among the
competing concerns of RDTs, namely (\rom{1}). The ease of programming
convergence, (\rom{2}) Latency, and (\rom{3}).  Staleness. While CRDTs
try to optimize for latency and staleness, they require a significant
amount of development and verification effort to be expended to ensure
convergence~\cite{kleppmann2017}. In contrast, \quark lets developers
derive convergent-by-construction RDTs from ordinary data data types
that are optimized for latency, but incur a staleness overhead that
delays the time to convergence. 




\section{Related Work}
\label{sec:related}

\balance
Mergeable Replicated Data Types (MRDTs) were introduced
in~\cite{mrdt}, where authors also demonstrate an approach for
deriving merge functions from first principles.  Despite resulting in
sensible merge semantics for several data types, their approach was
never shown to guarantee the convergence of the resultant MRDTs.
Indeed, as we demonstrate in Sec.~\ref{sec:motivation}, convergence
of unrestricted MRDT executions cannot be guaranteed due to the
presence of anomalies such as Fig.~\ref{fig:mrdt-exec-2} and
Fig.~\ref{fig:mrdt-exec-4}. \quark fixes this problem by extending
MRDTs with a runtime that limits their executions to those that are
guaranteed to converge.

% Concurrent Revisions~\cite{BBL+10} proposes versioned state for
% shared-memory concurrent programs in order to guarantee determinism.
% The fork-join concurrency they consider is however too restrictive for
% the asynchronous distributed setting where replicas do not have a
% single coherent view of the system state.  Furthermore the
% hierarchical organization of the concurrent actors with a designated
% \emph{server} is inconsistent with the fully-decentralized model of
% execution adopted by several collaborative applications.

State-centric replication was also explored in the context of
CRDTs~\cite{crdts}. However, such state-based CRDTs require the
replicated state to be organized as a lattice with the merge function
acting as as a least upper bound operator. We eliminate this
restriction in our setting with help of the \quark runtime. A thorough
comparison of our approach with the operation-based CRDTs can be found
in Secs.~\ref{sec:introduction} and~\ref{sec:motivation}.

Several verification techniques, program analyses, and tools have been
proposed to reason about the program behavior in a weakly-consistent
distributed setting~\cite{bailis-vldb, alvaro-calm,
gotsman-popl16, pldi15, redblue-osdi}. These techniques
treat replicated storage as a black box with a fixed pre-defined
consistency models. The focus is on assigning appropriate consistency
levels to operations so as to preserve application integrity. Such an
approach results in assigning sequential consistency (SC) to all
operations since the next weaker consistency model -- causal
consistency, is insufficient to guarantee convergence (as
Sec.~\ref{sec:motivation} demonstrates). Conversely, \quark cannot
reason about application-level invariants, such as $\C{balance}\ge 0$
in a banking application. Thus both approaches confer complimentary
benefits on application developers.

% Type systems
The meta-theory of \quark abstract machine
(Sec.~\ref{sec:abstract-sem} bears resemblance to the type safety
proofs carried out in Wright and Felleisen's style~\cite{WF92}.
Theorem~\ref{thm:progress} (Progress) is analogous to the Progress
lemma of type safety, and Theorem~\ref{thm:convergence} (Convergence)
is analogous to Preservation.  However, unlike the type-based
approaches, safety in \quark is enforced through run-time monitoring.
Static enforcement of convergence in MRDTs using, for e.g., Session
Types~\cite{HNMSession16}, is an interesting direction for future
work.

% Git
\quark's programming model is inspired by distributed version control
systems in general, and Git in particular. To the best of our
knowledge, operational semantics of Git has never been formalized.
This is not a serious concern considering that humans are heavily
involved in Git workflows, and the complexity of version histories
manifest by Git is bound by the limits of human cognition.
Nonetheless, it is possible to construct Git version histories that
result in non-convergent and counter-intuitive final
states~\cite{tycon,russell}. Such anomalies are more likely to occur
if Git version histories are manifest commensurately with distributed
executions of computer programs. \quark's run-time monitoring
pre-empts this possibility in our case.

Finally, the implementation of \quark bears resemblance to
Quelea~\cite{pldi15} as both systems offload low-level concerns to an
underlying data store. While Quelea enforces high-level invariants on
applications built with CRDTs that are assumed to be convergent,
\quark enforces convergence of data types that are otherwise not
guaranteed to converge.


\bibliography{all}

%% Acknowledgments
%\begin{acks}                            %% acks environment is optional
                                        %% contents suppressed with 'anonymous'
  %% Commands \grantsponsor{<sponsorID>}{<name>}{<url>} and
  %% \grantnum[<url>]{<sponsorID>}{<number>} should be used to
  %% acknowledge financial support and will be used by metadata
  %% extraction tools.
%\end{acks}


%% Bibliography
%\bibliography{bibfile}


%% Appendix
% \appendix
% \section{Appendix}

\end{document}
