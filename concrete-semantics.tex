\section{Concrete Semantics}
\label{sec:concrete-sem}

\begin{figure*}[t]
\begin{smathpar}
  \begin{array}{c}
    i,j\in\texttt{ReplicaIds}\spc
    b_i,b_j\in\texttt{Branches}\spc
    t\in\texttt{VersionVectors}\spc
    n\in\texttt{Values}\spc
%   \cedge,\fedge,\medge \in \texttt{Edges}\spc
%   G\in(\texttt{Vertices},\texttt{Edges})\\
    N : \texttt{VersionVector} \rightarrow \texttt{Value}\\
%   C : \texttt{Version} \rightarrow \Pow{\texttt{CommitId}} \spc
%   R : \texttt{ReplicaId} \rightarrow \texttt{Branch} \spc
    H : \texttt{ReplicaId} \rightarrow \texttt{Branch} \rightarrow \texttt{VersionVector} \spc
%   L : \texttt{Branch}\times\texttt{Branch} \rightarrow \texttt{Version}\\
    B : \texttt{ReplicaId} \rightarrow (\texttt{Branch} \times \mathbb{N}) \rightarrow \texttt{VersionVector} \spc
  \end{array}
\end{smathpar}
%
  \fbox {\( (B,N,H) \qstepsto (B',N',H')\)} 
%
%\bigskip

%
\begin{smathpar}
\begin{array}{c}
\RULE
{
  b_i\in dom(H_i)\spc
  t = H_i(b_i)\spc
  t' = t[b_i \mapsto t(b_i) + 1]\spc
}
{
  (B,N,H) \qstepsto (B_i[(b_i,t'(b_i)) \mapsto t'],\,
  N[t' \mapsto n],\, H_i[b_i \mapsto t'])
}
\spc
  [\rulelabel{Commit}]
\end{array}
\end{smathpar}
%

%
% \begin{smathpar}
% \begin{array}{c}
% \RULE
% {
%   b_i\in dom(H)\spc
%   b_j\not\in dom(H) \spc
%   t = H_i(b_i)\spc
%   t' = t[b_j \mapsto 1]
% }
% {
%   (B,N,H) \qstepsto (B_j[(b_j,1) \mapsto t'],\, N[t' \mapsto N(t)],\,
%         H_j[b_j \mapsto t'])
% }
% \spc
% [\rulelabel{Fork}]
% \end{array}
% \end{smathpar}
%

%
\begin{smathpar}
\begin{array}{c}
\RULE
{
  b_i,b_j\in dom(H_i)\spc
  \forall (k\in dom(H)).~b_k \in dom(H_i) \conj H_i(b_k) = H_k(b_k)\\
  t_i = H_i(b_i) \spc
  t_j = H_i(b_j)\spc
  t_l = t_i \sqcap t_j\spc
  t_l < t_i\spc
  t_l < t_j\spc
  t' = t_j \sqcup t_i[b_i \mapsto t_i(b_i)+1]\\
  \forall (k\in dom(H)).~
    (t_i \sqcap H_i(b_k)) \lesseqgtr (t_j \sqcap H_i(b_k))\spc
  n = {\sf merge}(N(t_l),\, N(H_i(b_i)),\, N(H_i(b_j))) 
}
{
  (B,N,H) \qstepsto (B_i[(b_i,t'(b_i)) \mapsto t'],
            N[t' \mapsto n],\, H_i[b_i \mapsto t'])
}
\spc
[\rulelabel{Merge}]
\end{array}
\end{smathpar}
%


%
% \begin{smathpar}
% \begin{array}{c}
% \RULE
% {
%   b_i,b_j\in dom(H_i)\spc
%   \forall (k\in dom(H)).~b_k \in dom(H_i) \conj H_i(b_k) = H_k(b_k)\\
%   t_i = H_i(b_i) \spc
%   t_j = H_i(b_j)\spc
%   t_i < t_j \spc
%   t' = t_j[b_i \mapsto t_i(b_i)+1]\spc
%   \forall (k\in dom(H)).~
%     (t_i \sqcap H_i(b_k)) \lesseqgtr (t_j \sqcap H_i(b_k))
%   %
% % \neg(H(b') \reaches H(b))\spc
% % H(b) \reaches H(b')\\
% }
% {
%   (B,N,H) \qstepsto (B_i[(b_i,t'(b_i)) \mapsto t'],
%             N[t' \mapsto N(t_j)],\, H_i[b_i \mapsto t'])
% }
% \spc
% [\rulelabel{FastFwd}]
% \end{array}
% \end{smathpar}
%



%
\begin{smathpar}
\begin{array}{c}
\RULE
{
  b_k\in dom(H_i)\spc
  b_k \in dom(H_j)\spc
  t = H_i(b_k) \spc
  t' = H_j(b_k)\spc
  t' > t
  %
% \neg(H(b') \reaches H(b))\spc
% H(b) \reaches H(b')\\
}
{
  (B,N,H) \qstepsto (B_i[(b_k,t(b_k)+n) \mapsto B_j(b_k,t(b_k)+n)
                          \,|\, n \in \{1,\ldots,t'(b_k) - t(b_k)\}],
            N,\, H_i[b_k \mapsto t'])
}
\spc
[\rulelabel{Sync}]
\end{array}
\end{smathpar}
%



\caption{The semantics of \quark distributed machine}
\label{fig:quark-semantics}
\end{figure*}


The \quark abstract machine of the previous section deliberately
ignores system-level concerns to focus on the semantics of
version-controlled state replication. In this section we present the
\quark \emph{distributed} machine that addresses the key system-level
concerns and provides a blueprint for a practical version-controlled
replicated state machine. We first reify the the one-to-one
correspondence between replicas and branches we assumed informally in
the previous section. This requires us to relax the assumption of the
state $\Delta$ being shared synchronously across all replicas. Next we
present an efficient method to track the version history with help
of vector clocks. Finally, we outline an algorithm for garbage
collecting older versions so that the version history need not grow
unboundedly.

Fig.~\ref{fig:quark-semantics} shows the operational semantics of the
distributed machine. The key difference from the abstract semantics of
Fig.~\ref{fig:git-semantics} is the presence of $\texttt{ReplicaId}$
indexing the components of the system state. We thus admit the
possibility of different replicas having different conceptions of the
state. Another major difference is the use of version
vectors~\cite{vectorclock} as identifiers and placeholders for
versions. A version vector of a version $v$ records the sequence
number of the last version from each branch that causally precedes $v$
(as per Def.~\ref{def:ancestor}).  Concretely, version vector is a map
from branches to natural numbers:
\begin{center}
  $\texttt{VersionVector}$ = $\texttt{Branch} \rightarrow \mathbb{N}$
\end{center}

The state of the distributed machine is the triple $\delta = (B,N,H)$.
Components $B$ and $H$ are indexed by $\texttt{ReplicaId}$ to let us
denote their replica-local copies. For instance, replica $i$'s copy of
the \emph{head map} $H$ is given by $H\;i$, which we abbreviate to
$H_i$ for notational convenience. Like $H$ in
Fig.~\ref{fig:git-semantics}, $H_i$ maps each branch to the version
vector of its head. Note that two replicas may be out of sync w.r.t
the information in $H$ and $B$.  For instance, $H_i(b)$ may not be
equal to $H_j(b)$ if replicas $i$ and $j$ are out of sync. On a
replica $i$, \emph{branch map} $B_i$ gives the version vector
corresponding to a particular sequence number on a branch. For
instance, the version vector of the first version on branch $b$ is
given by $B_i(b,1)$ on replica $i$. The value map $N$ maps version
vectors to their values. We assume a single copy of $N$ to simplify
the presentation. Generalizing Fig.~\ref{fig:quark-semantics} to allow
for replica-local copies of $N$ is straightforward.

Conspicuous by its absence in Fig.~\ref{fig:quark-semantics} is the
LCA map $L$, which we previously used to track the LCA version for
every pair of branches. The use of version vectors, coupled with the
unique LCA guarantee, obviates the need for an LCA map.  We can
instead identify the LCA of a pair of versions $v_1$ and $v_2$ by
computing the \emph{greatest lower bound} (GLB) of their version
vectors $t_1$ and $t_2$. Concretely:
\begin{center}
  $t_l \,=\, t_1 \sqcap t_2$
\end{center}
Where $t_l$ is the version vector of the LCA and $\sqcap$ is the GLB
operator. Conversely, the version vector indentifying the result of a
merge can be computed as the \emph{least upper bound} (LUB) of the two
version vectors involved in the merge. Concretely:
\begin{center}
  $t_m \,=\, t_1 \sqcup t_2$
\end{center}
Where $t_1$ and $t_2$, and $t_m$ are the version vectors of the
merging versions and the result of the merge, respectively. For
technical reasons, however, we increment the component of $t_m$
corresponding to the current branch $b$ to signify that this is a new
version on $b$. So the actual version vector of the merge is $t_m' =
t_m[b \mapsto t_m(b)+1]$. The GLB and LUB operations on version
vectors are standard~\cite{vectorclock} -- GLB is computed by taking
the component-wise minumum, and LUB by taking their maximum. The
comparison of version vectors is also standard -- $t_1 < t_2$ iff
$\forall b.~t_1(b) < t_2(b)$. Clearly, not all version vectors are
comparable. We write $t_1 \lesseqgtr t_2$ if vectors $t_1$ and $t_2$
are comparable.

\paragraph{Notation and Conventions} We enforce a one-to-one mapping
between replicas and branches by adopting the convention that branch
$b_i$ always corresponds to replica $i$. Concretely this means that
replica $i$ only ever creates new versions on branch $b_i$. Also,
replica $i$ can only update its local copies of $H$ and $B$, i.e.,
$H_i$ and $B_i$. For e.g., $H_i[b_i \mapsto t]$ updates the head of
branch $b_i$ to $t$ on replica $i$.  Since $H_i$ is simply an
abbreviation of $H\;i$ in the formalism, $H_i[b_i \mapsto t]$ actually
expands to $H_i[i \mapsto b_i \mapsto t]$. We exploit this notation in
Fig.~\ref{fig:quark-semantics}. 
% When two replicas, $i$ and $j$, are involved in an update, e.g.,
% \rulelabel{Fork}, we write $H_{\tuplee{i,j}}[b_j \mapsto t]$ to mean
% $H[i \mapsto b_j \mapsto t][j \mapsto b_j \mapsto t]$.

\paragraph{Rules} The transition relation $\qstepsto$ of the \quark
distributed machine is defined in Fig.~\ref{fig:quark-semantics}.
There is one rule corresponding to every rule in
Fig.~\ref{fig:git-semantics} (abstract semantics). \rulelabel{Commit}
describes replica $i$ committing a new version on the branch $b_i$
with the given value $n$.  The vector $t'$ for the new version is
obtained from that of the previous version $t$ by incrementing the
component $b_i$. The sequence number of the new version on $b_i$ is
$t'(b_i)$ and the branch map $B_i$ is updated to reflect that. Note
that for all $j \neq i$, $H_j$ and $B_j$ remain unchanged indicating
that other replicas are not (yet) aware of this commit.
%
\rulelabel{Fork} forks off a new branch $b_j$ from $b_i$. A new
replica $j$ is assumed to take over $b_j$. The version vector $t'$ of
the branch head now has a new component $b_j$ mapped to $1$ to denote
this is the first version on $b_j$. Map $B_j$ is updated accordingly.
Value $N(t')$ is same as its parent $N(t)$.

\rulelabel{Merge} describes the semantics of replica $i$ merging a
concurrent version from a (remote) replica $j$. The merging versions
are heads of their respective branches $b_i$ and $b_j$.  Like its
counterpart in Fig.~\ref{fig:git-semantics}, \rulelabel{Merge} insists
that the LCAs of every other branch $b_k$ with $b_i$ and $b_j$ be
causally related. This condition is however expressed in terms of
version vectors with help of the GLB ($\sqcap$) and comparision
($\lesseqgtr$) operators. For the LCA determination to be sound,
the merging replica $i$ needs to have an accurate conception of the
current version history. Futhermore, there cannot be a concurrent
merge happening elsewhere that undermines the judgment of replica $i$
(reg. the safety of $i \leftarrow j$ merge). These requirements are
enforced by the premise $\forall (k\in dom(H)).~b_k \in dom(H_i) \conj
H_i(b_k) = H_k(b_k)$, which insists that replica $i$'s knowledge of
every other branch $k$ be current. This condition effectively
linearizes merges by requiring a merge to either \emph{see} or
\emph{be seen} by every other merge. In practice this is achieved
through global coordination (Sec.~\ref{sec:implementation}). Note that
\rulelabel{Merge} only pre-empts a concurrent \rulelabel{Merge}, not a
concurrent \rulelabel{Commit}. A remote replica $k$ is allowed to keep
committing new versions on to $b_k$ even as it remains unaware of the
merge on replica $i$. Such leniency is imperative if the system were
to retain the performance benefits of asynchronous replication.
\rulelabel{FastFwd} is similar to \rulelabel{Merge}, so we don't
discuss it separately.

The rule \rulelabel{Sync} captures asynchronous communication between
a pair of replicas ($i$ and $j$) to get one of them ($i$) up-to-date
with the other ($j$).  Unlike the other rules, \rulelabel{Sync} does
not extend the version history graph, and therefore has no counterpart
in Fig.~\ref{fig:git-semantics}. It merely updates replica $i$'s
knowledge of branch $b_k$ if replica $j$ happens to have later updates
from $k$, i.e., $H_j(b_k)$ happens to be ahead of $H_i(b_k)$. The new
versions on $b_k$ known to $j$ but not $i$ are then simply transferred
to $i$.


