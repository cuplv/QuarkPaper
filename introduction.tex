\section{Introduction}

Programming geo-distributed replicated systems is hard given the complexity
of reasoning about constantly evolving states at different replicas.
Uncoordinated updates to replicas lead to states that are unlikely to
converge. A prominent approach to overcome this problem is to reorganize
the replicated state in terms of Commutative Replicated Data Types (CRDTs).
CRDTs require operations on the replicated state to commute, thereby
allowing such operations to be executed in different orders at replicas.
Applications composed of CRDTs therefore needn’t make assumptions about the
behavior of the underlying network; as long as the network eventually
delivers all in-flight messages, the replica states are guaranteed to
eventually converge. This guarantee is called Strong Eventual Consistency
(SEC). SEC is a sufficient condition to ensure eventual convergence of
replicated state. As such it provides a principled approach to build
distributed applications by composing CRDTs. It is however worthwhile to
note that SEC is not (proven to be) a necessary condition, and therefore
does not rule out the possibility of alternative approaches towards
convergent replicated state. Such an alternative approach is particularly
attractive if it frees application programming from the constrains of
commutativity.

The current work (SC MRDTs) proposes one such approach. The key idea behind
SC MRDTs is to decouple the orthogonal concerns of distributed applications
that are conflated by CRDTs, namely (1). Availability: the ability to
immediately apply updates to local replica without coordination, and (2).
Convergence: the ability to eventually apply updates at remote replicas in
a way that guarantees convergence. Note that while the Availability concern
explicitly rules out inter-replica coordination, Convergence doesn’t. In
other words, it is permissible for replicas to coordinate among themselves
to achieve convergent replication, as long as such coordination doesn’t
make a replica unavailable to its users.  SC MRDTs exploit this
observation. Like CRDTs, SC MRDTs allow updates from user-submitted
operations to be applied immediately to the local replica to minimize the
user-perceived latency. Unlike CRDTs however, SC MRDTs do not allow updates
to be applied at remote replica at arbitrary time and in arbitrary order.
Instead, the system underlying SC MRDTs carefully orchestrates the
distributed execution so as to merge the competing states at different
replicas in an order that guarantees convergence. Thus, any data type with
a merge operation qualifies to be a replicated data type without having to
re-engineer its operations around the commutativity principle. 
